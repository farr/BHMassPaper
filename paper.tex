\documentclass[preprint]{aastex}

%\usepackage{hyperref}
\usepackage{amssymb}
\usepackage{amsmath}

\newcommand{\fixme}[1]{[FIXME: #1]}
\newcommand{\Msun}{M_\odot}
\newcommand{\Mmin}{M_{\textnormal{min}}}
\newcommand{\Mmax}{M_{\textnormal{max}}}
\newcommand{\Nbin}{N_{\textnormal{bin}}}
\newcommand{\vtheta}{\vec{\theta}}
\newcommand{\order}[1]{\mathcal{O}\left( #1 \right)}

\bibliographystyle{hapj}

\begin{document}

\title{The Mass Distribution of Stellar-Mass Black Holes}

\author{Will M. Farr \and Niharika Sravan} 

\affil{Northwestern University Center for Interdisciplinary
  Exploration and Research in Astrophysics\\2145 Sheridan Rd.,
  Evanston, IL 60208}

\email{w-farr@northwestern.edu, niharika.sravan@gmail.com}

\author{Andrew Cantrell \and Laura Kreidberg \and Charles Bailyn}

\affil{Yale University Department of Astrophysics\\
  P.O. Box 208101, New Haven, CT 06520}

\email{andrew.cantrell@yale.edu, laura.kreidberg@yale.edu,
  charles.bailyn@yale.edu}

\author{Ilya Mandel\altaffilmark{1}}

\affil{MIT Kavli Institute, Cambridge, MA 02139}

\email{ilyamandel@chgk.info}

\and

\author{Vicky Kalogera} 

\affil{Northwestern University Center for Interdisciplinary
  Exploration and Research in Astrophysics\\2145 Sheridan Rd.,
  Evanston, IL 60208}

 \email{vicky@northwestern.edu}

 \altaffiltext{1}{Also: School of Physics and Astronomy, University of
   Birmingham, Edgbaston, Birmingham B15 2TT, UK}

\begin{abstract}
  We perform a Bayesian analysis of the mass distribution of
  stellar-mass black holes using the observed masses of 15 low-mass
  X-ray binary systems undergoing Roche lobe overflow and five
  higher-mass, wind-fed X-ray binary systems.  Using Markov Chain
  Monte Carlo calculations, we model the mass distribution both
  parametrically---as a power law, exponential, gaussian, combination
  of two gaussians, or log-normal distribution---and
  non-parametrically---as histograms with varying numbers of bins.  We
  provide confidence bounds on the shape of the mass distribution in
  the context of each model and compare the models with each other by
  calculating their relative Bayesian evidence as supported by the
  measurements.  The mass distribution of the low-mass systems is best
  fit by a power-law, while the distribution of the complete sample is
  best fit by the exponential model.  We examine the existence of a
  ``gap'' between the most massive neutron stars and least massive
  black holes, finding that the best model (the power law) fitted to
  the low-mass systems only gives a minimum black-hole mass of 4.3
  $\Msun$ (90\% confidence), while the best model (the exponential)
  fitted to all 20 systems gives a minumum black-hole mass of 4.5
  $\Msun$ (90\% confidence).  We conclude that our sample of black
  hole masses provides strong evidence of a gap between the maximum
  neutron star mass and minimum black hole mass.
\end{abstract}

\maketitle

\section{Introduction}
\label{sec:intro}

The most massive stars probably end their lives with a supernova
explosion, becoming stellar-mass black holes.  The mass distribution
of such black holes can provide important clues to the end stages of
evolution of these stars.  In addition, the mass distribution of
stellar-mass black holes is an important input in calculations of
rates of gravitational wave emission events from coalescing neutron
star-black hole and black hole-black hole binaries in the LIGO
gravitational wave observatory \citep{Abadie2010}.

Observations of low-mass X-ray binaries in both the optical and X-ray
can provide a measurement of the mass of the compact object in these
systems.  There are now enough such measurements where the mass of the
compact object is large enough to classify it as a black hole to
attempt to statistically reconstruct the mass distribution of these
stellar-mass black holes.

The first study of the mass distribution of stellar-mass black holes,
in \citet{Bailyn1998}, examined a sample of seven transient X-ray
binaries thought to contain a black hole, concluding in a Bayesian
analysis that the mass function was strongly peaked around seven solar
masses.  \citet{Bailyn1998} found evidence of a ``gap'' between the
least massive black hole and the upper limit for neutron star masses
of 3 $\Msun$ \citep{Kalogera1996}.  Such a gap is puzzling in light of
theoretical studies that predict a continuous distribution of compact
object supernova remnant masses with a smooth transition from neutron
stars to black holes \citep{Fryer2001}.

A more recent study \citep{Ozel2010}, also in a Bayesian framework,
largely confirmed these findings, examining 16 low-mass X-ray binary
systems containing black holes and finding a strongly peaked
distribution at $7.8 \pm 1.2 \, \Msun$.  \citet{Ozel2010} use two
models for the mass function: a Gaussian and a decaying exponential
with a minimum ``turn-on'' mass (motivated by the analytical model of
the black-hole mass function in \citet{Fryer2001}).  \citet{Ozel2010}
do not provide confidence limits for the minimum black hole mass,
instead discussing only the maximum-posterior parameters for their
models.  Nevertheless, it appears that their analysis confirms the
existence of a mass gap.  \citet{Ozel2010} discuss possible selection
effects that could lead to the appearance of a mass gap, but conclude
these effects could not produce the observed gap, which they therefore
claim is a real property of the black hole mass distribution.

In this work, we use observations of 20 galactic X-ray binaries that
contain black holes to constrain the mass function for stellar mass
black holes.  We consider separately 15 low-mass X-ray binaries
containing Roche-lobe-filling secondaries with masses smaller than the
black hole primary, and a sample including these systems and five
higher-mass, wind-fed systems where the secondary is heavier than the
black hole primary.

We use a Bayesian Markov-Chain Monte Carlo analysis to explore various
models for the black hole mass function for both samples.  We include
both parameteric models, such as a Gaussian, and non-parameteric
models where the mass function is represented by histograms with
various numbers of bins.  (Our set of models includes those of
\citet{Ozel2010} and \citet{Bailyn1998}.)  After computing posterior
distributions for the model parameters, we use model selection
techniques (including a new technique for efficient reversible-jump
MCMC \citep{Farr2010}) to compare the evidence for the various models
from both samples.

We find, like \citet{Ozel2010} and \citet{Bailyn1998}, strong evidence
for a mass gap among the best models for both systems, with the best
model for the lower-mass systems giving a minimum black hole mass of
4.3 $\Msun$ (at 90\% confidence), and the best model for the combined
sample of lower- and high-mass systems giving a minimum black hole
mass of 4.5 $\Msun$ (at 90\% confidence).  Among the models with lower
evidence, most also have a mass gap.

The theoretical model from \citet{Fryer2001}, a decaying exponential,
considered for the low-mass systems in \citet{Ozel2010}, is strongly
disfavored by our model selection.  We find that the lower-mass
systems are best described by a power law, followed closely by a
Gaussian (which is the other model from \citet{Ozel2010}).  We find
that the theoretical model from \citet{Fryer2001} is, however, the
preferred model for the combined sample of lower- and high-mass
systems.  A model with two separate Gaussian peaks also has high
evidence for the combined sample of systems.  While the failure of the
theoretically-motivated model to explain the lower-mass, Roche
lobe-filling systems suggests gaps in our understanding of the
formation of these systems, it is encouraging for population synthesis
studies \fixme{Vicky, do you have a couple of go-to papers for
  pop. synth we could cite here?} that the model of \citet{Fryer2001}
is a good fit to the full sample including both lower- and high-mass
systems.

\section{Systems}
\label{sec:systems}

The 20 X-ray binary systems on which this study is based are listed in
Table \ref{tab:sources}.  We separate the systems into 15 low-mass
systems in which the central black hole appears to be fed by
roche-lobe overflow from the secondary, and 5 higher-mass systems in
which the black hole is fed via winds (these systems all have a
secondary that appears to be more massive than the black hole).  The
low- and higher-mass systems undoubtedly have different evolutionary
tracks, and therefore it is reasonable that they would have different
black-hole mass distributions.  We will first analyze the 15 low-mass
systems alone (Section \ref{sec:results-lower-mass}), and then the
complete sample of 20 systems (Section \ref{sec:higher-mass}).

In each of these systems, spectroscopic measurements of the secondary
star provide an orbital period for the system and a semi-amplitude for
the secondary's velocity curve.  These measurements can be combined
into the mass function,
\begin{equation}
  \label{eq:mass-function}
  f(M) = \frac{P K^3}{2\pi G} = \frac{M \sin^3 i}{\left( 1 + q \right)^2},
\end{equation}
where $P$ is the orbital period, $K$ is the secondary's velocity
semi-amplitude, $M$ is the black hole mass, $i$ is the inclination of
the system, and $q \equiv M_2 / M$ is the mass ratio of the system.

The mass function defines a lower limit on the mass: $f(M) < M$.  To
accurately determine the mass of the black hole, the inclination $i$
and mass ratio $q$ must be measured.  Ideally, this can be
accomplished by fitting ellipsoidal light curves and study of the
rotational broadening of spectral lines from the secondary, but even
in the most studied case (see, e.g., \citet{Cantrell2010} on A0620)
this procedure is complicated.  In particular, contributions from an
accretion disk and hot spots in the disk can significantly distort the
measured inclination and mass ratios.  For some systems (e.g.\ GS 1354
\citep{Casares2009}) strong variability completely prevents
determination of the inclination from the lightcurve; in these cases
an upper limit on the inclination often comes from the observed lack
of eclipses in the lightcurve.  In general, accurately determining $q$
and $i$ requires a careful system-by-system analysis.

For the purposes of this paper, we adopt the following simplified
approach to the estimation of the black hole mass from the observed
data.  When an observable is well-constrained, we assume that the true
value is normally distributed about the measured value with a standard
deviation equal to the quoted observational error.  This is the case
for the mass function in all the systems we use, and for many systems'
mass ratios and inclinations.  When a large range is quoted in the
literature for an observable, we take the true value to be distributed
uniformly (for the mass ratio) or isotropically (for the inclination)
within the quoted range.  Table \ref{tab:sources} gives the assumed
distribution for the observables in the 20 systems we use.  We do not
attempt to deal with the systematic biases in the observational
determination of $f$, $q$, and $i$ in any realistic way; we are
currently investigating more realistic treatments of the errors
(including observational biases that can shift the peak of the true
mass distribution away from the ``best-fit'' mass in the
observations).  This treatment will appear in future work.

\begin{table}
  \begin{center}
    \begin{tabular}{|l|c|c|c|l|}
      \hline
      Source & $f$ ($\Msun$) & $q$ & $i$ (degrees) & References \\
      \hline \hline
      GRS 1915 & $N(9.5, 3.0)$ & $N(0.0857, 0.0284)$ & $N(70, 2)$ &
      \citet{Greiner2001} \\
      XTE J1118 & $N(6.44, 0.08)$ & $N(0.0264, 0.004)$ & $N(68, 2)$ &
      \citet{Gelino2008} \\ & & & & \citet{Harlaftis2005} \\
      XTE J1650 & $N(2.73, 0.56)$ & $U(0, 0.5)$ & $I(50, 80)$ &
      \cite{Orosz2004} \\
      GRS 1009 & $N(3.17, 0.12)$ & $N(0.137, 0.015)$ & $I(37, 80)$ &
      \cite{Filippenko1999} \\
      A0620 & $N(2.76, 0.036)$ & $N(0.06, 0.004)$ & $N(50.98, 0.87)$ &
      \citet{Cantrell2010} \\ & & & & \citet{Neilsen2008} \\
      GRO J0422 & $N(1.13, 0.09)$ & $U(0.076, 0.31)$ & $N(45, 2)$ &
      \citet{Gelino2003} \\
      Nova Mus 1991 & $N(3.01, 0.15)$ & $N(0.128, 0.04)$ & $N(54,1.5)$
      & \cite{Gelino2001} \\
      GRO J1655 & $N(2.73,0.09)$ & $N(0.3663, 0.04025)$ & $N(70.2,
      1.9)$ & \citet{Greene2001} \\
      4U 1543 & $N(0.25, 0.01)$ & $U(0.25, 0.31)$ & $N(20.7,1.5)$ & 
      \citet{Orosz2003} \\
      XTE J1550 & $N(7.73,0.4)$ & $U(0,0.04)$ & $N(74.7, 3.8)$ &
      \citet{Orosz2010} \\
      V4641 Sgr & $N(3.13,0.13)$ & $U(0.42,0.45)$ & $N(75,2)$ &
      \citet{Orosz2003} \\
      GS 2023 & $N(6.08, 0.06)$ & $U(0.056,0.063)$ & $I(66, 70)$ &
      \citet{Charles2006} \\
      & & & & \citet{Khargharia2010} \\
      GS 1354 & $N(5.73, 0.29)$ & $N(0.12,0.04)$ & $I(50, 80)$ & 
      \citet{Casares2009} \\
      Nova Oph 77 & $N(4.86,0.13)$ & $U(0, 0.053)$ & $I(60, 80)$ &
      \citet{Charles2006} \\
      GS 2000 & $N(5.01, 0.12)$ & $U(0.035, 0.053)$ & $I(43, 74)$ &
      \citet{Charles2006} \\
      \hline \hline
%%%%%%%%%%%%%%%%%%%%%%%%%%%%%%%%%%%%% High Mass Follows
      Cyg X1 & $N(0.251, 0.007)$ & $N(2.778, 0.386)$ & $I(23, 38)$ &
      \citet{Gies2003} \\ 
      M33 X7 & $N(0.46, 0.08)$ & $N(4.47, 0.61)$ & $N(74.6, 1)$ &
      \citet{Orosz2007} \\
      NGC 300 X1 & $N(2.6, 0.3)$ & $U(1.05, 1.65)$ & $I(60, 75)$ &
      \citet{Crowther2010} \\
      LMC X1 & $N(0.148, 0.004)$ & $N(2.91, 0.49)$ & $N(36.38, 2.02)$
      & \citet{Orosz2009} \\
      IC 10 X1 & $N(7.64, 1.26)$ & $U(0.7, 1.7)$ & $I(75, 90)$ & 
      \citet{Prestwich2007} \\
       & & & & \citet{Silverman2008} \\
       \hline
    \end{tabular}
  \end{center}
  \caption{\label{tab:sources} The source parameters for the 20 low
    mass X-ray binaries used in this work.  The first 15 systems have
    low-mass secondaries that feed the black hole via Roche lobe
    overflow; the last five systems have high-mass secondaries ($q
    \gtrsim 1$) that feed the black hole via winds.  In each line, $f$
    is the mass function for
    the compact object, $q$ is the mass ratio $M_2/M$, and $i$ is the
    inclination of the system to the line of sight.  We indicate the
    distribution used for the true parameters when computing the
    probability distributions for the masses of these systems:
    $N(\mu,\sigma)$ implies a Gaussian with mean $\mu$ and standard
    deviation $\sigma$, $U(a,b)$ is a uniform distribution between $a$ and
    $b$, and $I(\alpha,\beta)$ is an isotropic distribution between the
    angles $\alpha$ and $\beta$.}
\end{table}

From these assumptions, we can generate probability distributions for
the true mass of the black hole given the observations and errors via
the Monte Carlo method: drawing samples of $f$, $q$, and $i$ from the
assumed distributions and computing the mass implied by Equation
\eqref{eq:mass-function} gives samples of $M$ from the distribution
induced by the relationship in Equation \eqref{eq:mass-function}.
Mass distributions generated in this way for the systems used in this
work are shown in Figure \ref{fig:low-masses} and Figure
\ref{fig:high-masses}.  Systems for which $i$ is poorly constrained
have broad ``tails'' on their mass distributions.  These mass
distributions constitute the ``observational data'' we will use in the
remainder of this paper.

\begin{figure}
  \begin{center}
    \plotone{plots/all-masses} 
  \end{center}

  \caption{\label{fig:low-masses} The mass distributions implied by
    Equation \eqref{eq:mass-function} and the assumed distributions on
    observational parameters given in Table \ref{tab:sources} for the
    lower-mass sources.  The significant asymmetry and long tails in
    many of these distributions are the result of the non-linear
    relationship (Equation \eqref{eq:mass-function}) between $M$, $f$,
    $q$, and $i$.}
\end{figure}

\begin{figure}
  \begin{center}
    \plotone{plots/high-masses}
  \end{center}
  \caption{\label{fig:high-masses} Mass distributions for the
    wind-fed, higher-mass systems computed from the distributions on
    observed data in Table \ref{tab:sources} using Equation
    \eqref{eq:mass-function}.  The asymmetry and long tails in these
    distributions are the result of the non-linear relationship
    between $M$, $f$, $q$, and $i$.}
\end{figure}

\section{Statistical Analysis}
\label{sec:models}

In this section we describe the statistical analysis we will apply to
various models for the underlying mass distribution from which the
low-mass systems in Table \ref{tab:sources} were drawn.  Results of
the analysis appear in Section \ref{sec:results}.

\subsection{Bayesian Inference}

The end result of our statistical analysis will be the probability
distribution for the parameters of each model implied by the data from
Section \ref{sec:systems} in combination with our prior assumptions
about the probability distribution for the parameters.  Bayes' rule
relates these quantities.  For a model with parameters $\vtheta$ in
the presence of data $d$, Bayes' rule states
\begin{equation}
  \label{eq:Bayes-rule}
  p(\vtheta | d) = \frac{p(d | \vtheta) p(\vtheta)}{p(d)}.
\end{equation}
Here, $p(\vtheta|d)$, called the posterior probability distribution
function, is the probability distribution for the parameters $\vtheta$
implied by the data $d$; $p(d|\vtheta)$, called the likelihood, is the
probability of observing data $d$ given that the model parameters are
$\vtheta$; $p(\vtheta)$, called the prior, reflects our estimate of the
probability of the various model parameters in the absence of any
data; and $p(d)$, called the evidence, is an overall normalizing
constant ensuring that
\begin{equation}
  \int d\theta\, p(\vtheta|d) = 1,
\end{equation}
whence
\begin{equation}
  \label{eq:evidence-def}
  p(d) = \int d\vtheta\, p(d|\vtheta) p(\vtheta).
\end{equation}

In our context, the data are the mass distributions given in Section
\ref{sec:systems}: $d = \{ p_i(M)| i = 1, 2, \ldots, 20 \}$.  We
assume that the measurements in Section \ref{sec:systems} are
independent, so the complete likelihood is given by a product of the
likelihoods for the individual measurements.  For a model with
parameters $\vtheta$ that predicts a mass distribution $p(M|\vtheta)$
for black holes, we have
\begin{equation}
  \label{eq:likelihood-def}
  p(d|\vtheta) = \prod_i \int dM\, p_i(M) p(M|\vtheta).
\end{equation}
That is, the likelihood of an observation is the average over the
individual mass distribution implied by the observation, $p_i(M)$, of
the probability for a black hole of that mass to exist according to
the model of the mass distribution, $p(M | \vtheta)$.  We approximate
the integrals as averages of $p(M|\vtheta)$ over the Monte Carlo mass
samples drawn from the distributions in Table \ref{tab:sources} (also
see Figures \ref{fig:low-masses} and \ref{fig:high-masses}):
\begin{equation}
  p(d|\vtheta) \approx \prod_i \frac{1}{N_i} \sum_{j = 1}^{N_i} p(M_{ij} | \vtheta),
\end{equation}
where $M_{ij}$ is the $j$th sample (out of a total $N_i$) from the
$i$th individual mass distribution.

Our calculation of the likelihood of each observation does not include
any attempt to account for selection effects in the observations.  We
simply assume (almost certainly incorrectly) that any black hole drawn
from the underlying mass distribution is equally likely to be
observed.  The results of \citet{Ozel2010} suggest that selection
effects are unlikely to significantly bias our analysis.

For a mass distribution with several parameters, $p(\vtheta | d)$
lives in a multi-dimensional space.  Previous works
\citep{Ozel2010,Bailyn1998} have considered models with only two
parameters, evaluating $p(\vtheta|d)$ on a grid.  Many of our models
for the underlying mass distribution have three or more parameters.
Exploring the entirety of these parameter spaces with a grid rapidly
becomes prohibitive as the number of parameters increases.  A more
efficient way to explore the distribution $p(\vtheta | d)$ is to use a
Markov Chain Monte Carlo (MCMC) method.  MCMC methods produce a chain
(sequence) of parameter samples, $\{ \vtheta_i \, | \, i = 1, \ldots
\}$, such that a particular parameter sample, $\vtheta$, appears in
the sequence with a frequency equal to its posterior probability,
$p(\vtheta|d)$.  In this way, regions of parameter space where
$p(\vtheta|d)$ is large are sampled densely while regions where
$p(\vtheta|d)$ is small are effectively ignored.

Once we have a chain of samples from $p(\vtheta|d)$, the distribution
for any quantity of interest can be computed by evaluating it on each
sample in the chain and forming a histogram of these values.  For
example, to compute the one-dimensional distribution for a single
parameter obtained by integrating over all other dimensions in
parameter space, called the ``marginalized'' distribution, one plots
the histogram of the values of that parameter appearing in the chain.

\subsection{Priors}
\label{sec:priors}

An important part of any Bayesian analysis is the priors placed on the
parameters of the model.  The choice of priors can bias the results of
the analysis through both the shape and the range of prior support in
parameter space.  The prior should reflect the ``best guess'' for the
distribution of parameters before examining any of the data.  In the
absence of any information about the distribution of parameters, it is
best to choose a prior that is broad and uniformative to avoid biasing
the posterior as much as possible.

A prior that is independent of parameters, $\vtheta$, in some region,
called ``flat,'' results in a posterior that is proportional to the
likelihood (see Equation \eqref{eq:Bayes-rule}).  A flat prior does
not change the shape of the posterior.  However, the choice of a flat
prior is parameterization-dependent: a change of parameter from
$\vtheta$ to $\vtheta' = \vec{f}(\vtheta)$ can change a flat
distribution into one with non-trivial structure.  In this work, we
choose priors that are flat when the parameters are measured in
physical units.  In particular, for the log-normal model (Section
\ref{sec:log-normal}) the natural parameters for the distribution are
the mean, $\langle \log M \rangle$, and standard deviation,
$\sigma_{\log M}$, in $\log M$, but we choose priors that are flat in
$\langle M \rangle$ and $\sigma_M$.

The range of prior support can also affect the results of a Bayesian
analysis.  Because priors are normalized, prior support over a larger
region of parameter space results in a smaller prior probability at
each point.  Such ``wide'' priors are implicitly claiming that any
particular sub-region of parameter space is less likely than it would
be under a prior of the same shape but smaller support volume.  This
difference is important in model selection (Section
\ref{sec:model-selection}): when comparing two models with the same
likelihood, one with wide priors will seem less probable than one with
narrower priors.  Of course, priors should be wide enough to encompass
all regions of parameter space that have significant likelihood.  To
make the model comparison in Section \ref{sec:model-selection} fair,
we choose prior support in parameter space so that the allowed
parameter values for each model give distributions for which nearly
all the probability lies in the range $0 \leq M \leq 40 \Msun$.

\subsection{Parametric Models for the Black Hole Mass Distribution}
\label{sec:parametric-models}

Here we discuss the various parametric models of the underlying black
hole mass distribution considered in this paper.

\subsubsection{Power-Law Models}
\label{sec:power-law}

Many astrophysical distributions are power laws.  Let us assume that
the BH mass distribution is given by
\begin{equation}
  \label{eq:power-law-dist}
  p(M|\vtheta) = p(M|\{\Mmin, \Mmax, \alpha \}) =
  \begin{cases}
    A M^\alpha & \Mmin \leq m \leq \Mmax \\
    0 & \textnormal{otherwise}
  \end{cases}.
\end{equation}
The normalizing constant $A$ is 
\begin{equation}
  A = \frac{1+\alpha}{\Mmax^{1+\alpha} - \Mmin^{1+\alpha}}.
\end{equation}
We choose uniform priors on $\Mmin$ and $\Mmax \geq \Mmin$ between 0 and
$40 \Msun$, and uniform priors on the exponent $\alpha$ in a broad
range between $-15$ and $13$:
\begin{equation}
  p(\vtheta) = p(\{\Mmin, \Mmax, \alpha\}) = 
  \begin{cases}
    2 \frac{1}{40^2}\frac{1}{28} & 0 \leq \Mmin \leq \Mmax
    \leq 40, \quad -15 \leq \alpha \leq 13 \\
    0 & \textnormal{otherwise}
  \end{cases}.
\end{equation}

Our MCMC analysis output is a list of $\{\Mmin, \Mmax, \alpha\}$
values distributed according to the posterior 
\begin{equation}
  p(\vtheta|d) = p(\{\Mmin, \Mmax, \alpha\}|d) \propto p(d|\{\Mmin,
  \Mmax, \alpha\}) p(\{\Mmin, \Mmax, \alpha\}),
\end{equation}
with the likelihood $p(d|\{\Mmin, \Mmax, \alpha\})$ defined in
Equation \eqref{eq:likelihood-def}.  

\subsubsection{Decaying Exponential}
\label{sec:exponential}

\citet{Fryer2001} studied the relation between progenitor and remnant
mass in simulations of supernova explosions.  Combining this with the
mass function for supernova progenitors, they suggested that the
black-hole mass distribution may be well-represented by a decaying
exponential with a minimum mass: \fixme{Vicky, do you want to say
  anything more here?}
\begin{equation}
  \label{eq:exp-def}
  p(M|\vtheta) = p(M|\{\Mmin, M_0\}) = 
  \begin{cases}
    \frac{e^{\frac{\Mmin}{M_0}}}{M_0} \exp \left[ - \frac{M}{M_0}
    \right] & M \geq \Mmin \\
    0 & \textnormal{otherwise}
  \end{cases}.
\end{equation}
We choose a prior for this model where $\Mmin$ is uniform between 0
and $40 \Msun$.  For each $\Mmin$, we choose $M_0$ uniformly within a
range ensuring that $40\Msun$ is at least two scale masses above the
cutoff: $40\Msun \geq \Mmin + 2 M_0$.  This ensures that the majority
of the mass probability lies in the range $0 \leq M \leq 40\Msun$.
The resulting prior is
\begin{equation}
  p(\vtheta) = p(\{\Mmin, M_0\}) = 
  \begin{cases}
    \frac{4}{40^2} & 0 \leq \Mmin \leq 40, \quad 0 < M_0, \quad \Mmin
    + 2 M_0 \leq 40, \\
    0 & \textnormal{otherwise}
  \end{cases}
\end{equation}

\subsubsection{Gaussian and Two-Gaussian Models}
\label{sec:gaussian}

The mass distributions in Figure \ref{fig:low-masses} all peak in a
relatively narrow range near $\sim 10 \Msun$.  The prototypical
single-peaked probability distribution is a Gaussian:
\begin{equation}
  \label{eq:gaussian-def}
  p(M|\vtheta) = p(M|\{\mu, \sigma\}) = \frac{1}{\sigma \sqrt{2\pi}}
  \exp\left[ - \left(\frac{M - \mu}{\sqrt{2} \sigma} \right)^2 \right].
\end{equation}
We use a prior on the mean mass, $\mu$, and the standard deviation,
$\sigma$, that ensures that the majority of the mass distribution lies
below $40 \Msun$:
\begin{equation}
  \label{eq:gaussian-prior-def}
  p(\{\mu,\sigma\}) = 
  \begin{cases}
    \frac{8}{40^2} & 0 \leq \mu \leq 40, \quad \sigma \geq 0, \quad
    \mu + 2\sigma \leq 40 \\
    0 & \textnormal{otherwise}
  \end{cases},
\end{equation}
where both $\mu$ and $\sigma$ are measured in solar masses.  

Though we do not expect to find a second peak in the low-mass
distribution, we may find evidence of one when exploring the combined
low- and higher-mass samples.  To look for a second peak in the
black-hole mass distribution, we use a two-Gaussian model:
\begin{multline}
  \label{eq:two-gaussian-def}
  p(M|\vtheta) = p(M|\{\mu_1, \mu_2, \sigma_1, \sigma_2, \alpha\}) = \\
  \frac{\alpha}{\sigma_1 \sqrt{2\pi}} \exp\left[ - \left( \frac{M -
        \mu_1}{\sqrt{2}\sigma_1} \right)^2 \right] + \frac{1-\alpha}{\sigma_2 \sqrt{2\pi}} \exp\left[ - \left( \frac{M -
        \mu_2}{\sqrt{2}\sigma_2} \right)^2 \right].
\end{multline}
The probability is a linear combination of two Gaussians with weight
$\alpha$.  We restrict $\mu_1 < \mu_2$ and also impose combined
conditions on $\mu_i$ and $\sigma_i$ that ensure that most of the mass
probability lies below $40 \Msun$ with the prior 
\begin{equation}
  p(\{\mu_1, \mu_2, \sigma_1, \sigma_2, \alpha\}) = 
  \begin{cases}
    2 p(\{\mu_1, \sigma_1\}) p(\{\mu_2, \sigma_2\}) & \mu_1 \leq
    \mu_2, \quad 0 \leq \alpha \leq 1 \\
    0 & \textnormal{otherwise}
  \end{cases},
\end{equation}
where the single-Gaussian prior, $p(\{\mu_i, \sigma_i\})$, is defined
in Equation \eqref{eq:gaussian-prior-def}.

\subsubsection{Log Normal}
\label{sec:log-normal}

Many of the mass distributions for the systems in Figure
\ref{fig:low-masses} rise rapidly to a peak and then fall off more
slowly in a longer tail toward high masses.  So far, none of the
parameterized distributions we have discussed have this property.  In
this section, we consider a log-normal model for the underlying mass
distribution; the log-normal distribution has a rise to a peak with a
slower falloff in a long tail.  

The log-normal distribution gives $\log M$ a Gaussian distribution
with mean $\mu$ and standard deviation $\sigma$:
\begin{equation}
  \label{eq:log-normal-def}
  p(M|\vtheta) = p(M|\{\mu, \sigma \}) = \frac{1}{
    \sqrt{2\pi} M \sigma} \exp\left[ -\frac{\left(\log M - \mu\right)^2}{2 \sigma^2} \right].
\end{equation}
The parameters $\mu$ and $\sigma$ are dimensionless; the mean mass
$\langle M \rangle$ and mass standard deviation $\sigma_M$ are related
to $\mu$ and $\sigma$ by
\begin{eqnarray}
  \label{eq:avg-M}
  \langle M \rangle & = & \exp\left( \mu + \frac{1}{2} \sigma^2
  \right) \\
  \label{eq:sigma-M}
  \sigma_M & = & \langle M \rangle \sqrt{\exp\left( \sigma^2 \right) - 1}.
\end{eqnarray}
For a fair comparison with the other models, we impose a prior that is
flat in $\langle M \rangle$ and $\sigma_M$.  To ensure that most of
the probability in this model occurs for masses below $40 \Msun$, we
require $\langle M \rangle + 2 \sigma_M \leq 40$, resulting in a
prior
\begin{equation}
  p(\vtheta) = p(\{\mu,\sigma\}) = 
  \begin{cases}
    \frac{4}{40^2} \left| \frac{\partial \left(\langle M \rangle,
          \sigma_M \right)}{\partial \left( \mu, \sigma \right)}
    \right| & \sigma > 0, \quad \langle M \rangle + 2 \sigma_M \leq 40
    \\
    0 & \textnormal{otherwise}
  \end{cases},
\end{equation}
where 
\begin{equation}
  \left| \frac{\partial \left(\langle M \rangle,
          \sigma_M \right)}{\partial \left( \mu, \sigma \right)}
    \right| = \frac{\exp\left( 2 \left( \mu + \sigma^2 \right)\right)
      \sigma}{\sqrt{\exp\left( \sigma^2 \right) - 1}}
\end{equation}
is the determinant of the Jacobian of the map in Equations
\eqref{eq:avg-M} and \eqref{eq:sigma-M}.

\subsection{Non-Parametric Models for the Black Hole Mass Distribution}
\label{sec:non-parametric-models}

The previous subsection discussed models for the underlying black hole
mass distribution that assumed particular parameterized shapes for the
distribution.  In this subsection, we will discuss models that do not
assume a priori a shape for the black hole mass distribution.  The
fundamental non-parametric distribution in this section is a
histogram with some number of bins, $\Nbin$.  Such a distribution is
piecewise-constant in $M$.

One choice for representing such a histogram would be to fix the bin
locations, and allow the heights to vary.  With this approach, one
should be careful not to ``split'' features of the mass distribution
across more than one bin in order to avoid diluting the sensitivity to
such features; similarly, one should avoid including more than ``one''
feature in each bin.  The locations of the bins, then, are crucial.
An alternative representation of histogram mass distributions avoids
this difficulty.

We choose to represent a histogram mass distribution with $\Nbin$ bins
by allocating a fixed probability, $1/\Nbin$, to each bin.  The lower
and upper bounds for each bin are allowed to vary; when these are
close to each other (i.e.\ the bin is narrow), the distribution will
have a large value, and conversely when the bounds are far from each
other.  We assume that the non-zero region of the distribution is
contiguous, so we can represent the boundaries of the bins as a
non-decreasing array of masses, $w_0 \leq w_1 \leq \ldots \leq
w_{\Nbin}$, with $w_0$ the minimum and $w_{\Nbin}$ the maximum mass
for which the distribution has support.  This gives the distribution
\begin{equation}
  \label{eq:hist-def}
  p(M|\theta) = p(M|\{w_0, \ldots, w_{\Nbin}\}) = 
  \begin{cases}
    0 & M < w_0 \textnormal{ or } w_{\Nbin} \leq M \\
    \frac{1}{\Nbin} \frac{1}{w_{i+1} - w_i} & w_i \leq M < w_{i+1}
  \end{cases}.
\end{equation}

For priors on the histogram model with $\Nbin$ bins, we assume that
the bin boundaries are uniformly distributed between 0 and $40 \Msun$
subject only to the constraint that the boundaries are non-decreasing
from $w_0$ to $w_{\Nbin}$:
\begin{equation}
  p(\{w_0, \ldots, w_{\Nbin}\}) = 
  \begin{cases}
    \frac{\left(\Nbin+1\right)!}{40^{\Nbin+1}} & 0 \leq w_0 \leq w_1
    \leq \ldots \leq w_{\Nbin} \leq 40 \\
    0 & \textnormal{otherwise}
  \end{cases}.
\end{equation}

We consider histograms with up to five bins in this work.  We will see
that the evidence for the histogram models (see Sections
\ref{sec:model-selection}, \ref{sec:lower-mass-model-selection}, and
\ref{sec:high-mass-model-selection}) from both the lower-mass and
complete datasets is decreasing as the number of bins reaches five,
indicating that increasing the number of bins beyond five would not
sufficiently improve the fit to the mass distribution to compensate
for the extra parameter-space volume implied by the additional
parameters.

\subsection{Bayesian Model Selection}
\label{sec:model-selection}

In Sections \ref{sec:parametric-models} and
\ref{sec:non-parametric-models}, we discussed a series of models for
the underlying black hole mass distribution.  Our MCMC analysis will
provide the posterior distribution of the parameters within each
model, but does not tell us which models are more likely to correspond
to the actual distribution.  This model selection problem is the topic
of this section.

Consider a set of models, $\{M_i| i = 1, \ldots\}$, each with
corresponding parameters $\vtheta_i$.  Re-writing Equation
\eqref{eq:Bayes-rule} to be explicit about the assumption of a
particular model, we have
\begin{equation}
  p(\vtheta_i | d, M_i) = \frac{p(d|\vtheta_i, M_i) p(\vtheta_i | M_i)}{p(d|M_i)}.
\end{equation}
This gives the posterior probability of the parameters $\vtheta_i$ in
the context of model $M_i$.  But, the model itself can be regarded as
a discrete parameter in a larger ``super-model'' that encompasses all
the $M_i$.  The parameters for the super-model are $\{M_i,
\vtheta_i\}$: a choice of model and the corresponding parameter value
within that model.  Each point in the super-model parameter space is a
statement that, e.g., ``the underlying mass distribution is a
Gaussian, with parameters $\mu$ and $\sigma$,'' or ``the underlying
mass distribution is a triple-bin histogram with parameters $w_1$,
$w_2$, $w_3$, and $w_4$,'' or ....  The posterior probability of the
super-model parameters is given by Bayes' rule:
\begin{equation}
  \label{eq:bayes-explicit-model}
  p(\vtheta_i, M_i|d) = \frac{p(d|\vtheta_i, M_i) p(\vtheta_i |M_i) p(M_i)}{p(d)},
\end{equation}
where we have introduced the model prior $p(M_i)$, which represents
our estimate on the probability that model $M_i$ is correct in the
absence of the data $d$.  The normalizing evidence is now
\begin{equation}
  \label{eq:multi-model-evidence-def}
  p(d) = \sum_i \int d\vtheta_i\, p(\vtheta_i, M_i|d) = \sum_i
  p(d|M_i) p(M_i),
\end{equation}
writing the single-model evidence from Equation
\eqref{eq:evidence-def} as $p(d|M_i)$ to be explicit about the
dependence on the choice of model.

To compare the various models $M_i$, we are interested in the
marginalized posterior probability of $M_i$:
\begin{equation}
  \label{eq:model-posterior-def}
  p(M_i|d) \equiv \int d\vtheta_i\, p(\vtheta_i, M_i|d).
\end{equation}
This is the integral of the posterior over the entire parameter space
of model $M_i$.  The marginalized posterior probability of model $M_i$
can be re-written in terms of the single-model evidence, $p(d|M_i)$
(see Equations \eqref{eq:bayes-explicit-model} and
\eqref{eq:evidence-def}):
\begin{equation}
  \label{eq:model-evidence-def}
  p(M_i|d) = \int d\vtheta_i\, p(\vtheta_i, M_i|d) = \frac{p(M_i)}{p(d)} \int d\vtheta_i
  p(d|\vtheta_i,M_i) p(\vtheta_i|M_i) = \frac{p(d|M_i) p(M_i)}{p(d)}.
\end{equation}

Here and throughout, we assume that any of the models in Section
\ref{sec:models} are equally likely a priori, so the model priors are
equal:
\begin{equation}
  p(M_i) = \frac{1}{N_{\textnormal{model}}}.
\end{equation}

A powerful technique%
\footnote{We also attempted to compute $p(M_i|d)$ using two other
  methods: the well-known harmonic-mean estimator and the direct
  integration methods described in \citet{Weinberg2010}.  The harmonic
  mean is known to be very sensitive to outlying points in the MCMC in
  general, and we found this to be true in our specific application.
  The statistical properties of the direct integration algorithm from
  \citet{Weinberg2010} are less certain, but we found that it was
  quite noisy in our application compared to the reversible-jump MCMC.
  Due to the statistical noise in the other two methods, we use the
  results from our reversible jump MCMC analysis for model
  selection.} %
for computing $p(M_i|d)$ is the reversible-jump MCMC
\citep{Green1995}.  Reversible jump MCMC, discussed in more detail in
Appendix \ref{sec:reversible-jump-mcmc}, is a standard MCMC analysis
conducted in the super-model.  The result of a reversible jump MCMC is
a chain of samples, $\{ M_i, \vtheta_i\, | \, i = 1, \ldots \}$, from the
super-model parameter space.  The integral in Equation
\eqref{eq:model-evidence-def} can be estimated by counting the number
of times that a given model $M_i$ appears in the reversible jump MCMC
chain:
\begin{equation}
  p(M_i|d) = \int d\vtheta_i p(M_i, \vtheta_i|d) \approx \frac{N_i}{N},
\end{equation}
where $N_i$ is the number of MCMC samples that have discrete parameter
$M_i$, and $N$ is the total number of samples in the MCMC. 

Naively implemented reversible jump MCMCs can be very inefficient when
the posteriors for a model or models are strongly peaked.  In this
circumstance, a proposed MCMC jump into one of the peaked models is
unlikely to land on the peak by chance; since it is rare to propose a
jump into the important regions of parameter space of the peaked model
in a naive reversible jump MCMC, the output chain must be very long to
ensure that all models have been compared fairly.  We describe a new
algorithm in Appendix \ref{sec:reversible-jump-mcmc} that produces
very efficient jump proposals for a reversible jump MCMC by exploiting
the information about the model posteriors we have from the
single-model MCMC samples.  (See also \citet{Farr2010}.)  With this
algorithm, reasonable chain lengths can fairly compare all the models
under consideration.  We have used this algorithm to perform 10-way
reversible jump MCMCs to calculate the relative evidence for both the
parametric and non-parametric models in this study.  These results
appear in Section \ref{sec:results}.

\section{Results}
\label{sec:results}

In this section we discuss the results of our MCMC analysis of the
posterior distributions of parameters for the models in Sections
\ref{sec:parametric-models} and \ref{sec:non-parametric-models}.  We
also discuss model selection results.  The results in Section
\ref{sec:results-lower-mass} apply to the lower-mass sample of
systems, while those of Section \ref{sec:higher-mass} apply to the
complete sample of systems.

\subsection{Lower-Mass Systems}
\label{sec:results-lower-mass}

Table \ref{tab:low-mass-parametric} gives quantiles of the
marginalized parameter distributions of the parametric models implied
by the low-mass data.  Table \ref{tab:low-mass-non-parametric} gives
the quantiles of the histogram bin boundaries in the non-parametric
analysis implied by the low-mass data.

\begin{table}
  \begin{center}
    \begin{tabular}{|l|c|c|c|c|c|c|}
      \hline
      Model & Parameter & 5\% & 15\% & 50\% & 85\% & 95\% \\
      \hline \hline
      Power Law (Equation \eqref{eq:power-law-dist}) & $\Mmin$ & 
      1.2786 &  4.1831 &  6.1001 &  6.5011 &  6.6250 \\
      \hline
       & $\Mmax$ & 8.5578 &  8.9214 & 23.3274 & 36.0002 & 38.8113 \\
       \hline
       & $\alpha$ & -12.4191 & -10.1894 & -6.3861 &  2.8476 &  5.6954 \\
       \hline \hline
       Exponential (Equation \eqref{eq:exp-def}) & $\Mmin$ & 
       5.0185 &  5.4439 &  6.0313 &  6.3785 &  6.5316 \\
       \hline
       & $M_0$ & 0.7796 &  0.9971 & 1.5516 &  2.4635 &  3.2518 \\
       \hline \hline
       Gaussian (Equation \eqref{eq:gaussian-def}) & $\mu$ & 
       6.6349 &  6.9130 &  7.3475 & 7.7845 & 8.0798 \\
       \hline
       & $\sigma$ & 0.7478 &  0.9050  & 1.2500 &  1.7335 & 2.1134 \\
       \hline \hline
       Two Gaussian (Equation \eqref{eq:two-gaussian-def}) & $\mu_1$ & 
       5.4506 &  6.3877 &  7.1514 &  7.6728  & 7.9803 \\
       \hline
       & $\mu_2$ & 7.2355 &  7.7387 & 12.3986 & 25.2456 & 31.4216 \\
       \hline
       & $\sigma_1$ & 0.3758 &  0.7626 &  1.2104 &  1.7981 &  2.3065 \\
       \hline
       & $\sigma_2$ & 0.2048 & 0.6421 & 1.9182 &  5.2757  & 7.2625 \\
       \hline
       & $\alpha$ & 0.0983 &  0.3526 & 0.8871 &  0.9792 &  0.9936 \\
       \hline \hline
       Log Normal (Equation \eqref{eq:log-normal-def}) & $\langle M \rangle$ & 
       6.7619 &  7.0122 &  7.4336  &  7.9159  &  8.2942 \\
       \hline 
       & $\sigma_M$ & 0.7292  &  0.8920  & 1.2704  &  1.8695  &  2.4069 \\
       \hline
    \end{tabular}
  \end{center}
  \caption{\label{tab:low-mass-parametric} Quantiles of the
    marginalized distribution for each of the parameters in the models
    discussed in Section \ref{sec:parametric-models} implied by the low-mass data.  We indicate
    the 5\%, 15\%, 50\% (median), 85\%, and 95\% quantiles.  The
    marginalized distribution can be misleading when there are strong
    correlations between variables.  For example, while the
    marginalized distributions for the power law parameters are quite
    broad, the distribution of mass distributions implied by the power
    law MCMC samples is similar to the other models.  This occurs in
    spite of the broad marginalized distributions because of the
    correlations between the slope and limits of the power law
    discussed in Section \ref{sec:power-law}.}
\end{table}

Recall that each MCMC sample in our analysis gives the parameters for
a model of the black hole mass distribution.  The chain of samples of
parameters for a particular model gives us a distribution of black
hole mass distributions.  Figure \ref{fig:dists} gives a sense of the
shape and range of the distributions of black hole mass distributions
that result from our MCMC analysis.  In Figure \ref{fig:dists} we plot
the median, 10\% and 90\% values of the black hole mass distributions
that result from the MCMC chains.  Because the choice of parameters
that gives, for example, the median distribution value at one mass
need not give the median distribution at another mass, these curves do
not necessarily look like the underlying model for the mass
distribution.  For the same reason, they are not necessarily
normalized.

\begin{figure}
  \begin{center}
    \plotone{plots/dist}
  \end{center}
  \caption{\label{fig:dists} The median (solid line), 10\% (lower
    dashed line) and 90\% (upper dashed line) values of the black hole
    mass distribution, $p(M|\theta)$, at various masses implied by the
    posterior $p(\theta|d)$ for the models discussed in Sections
    \ref{sec:parametric-models} and \ref{sec:non-parametric-models}.
    These distributions use only the 15 low-mass observations in Table
    \ref{tab:sources} (the complete sample is analyzed in Section
    \ref{sec:higher-mass}).  Note that these ``distributions of
    distributions'' are not necessarily normalized, and need not be
    shaped like the underlying model distributions.}
\end{figure}

\subsubsection{Power Law}

In Figure \ref{fig:power-law} , we display a histogram of the
resulting samples in each of the parameters $\Mmin$, $\Mmax$, and
$\alpha$ for the power law model (see Equation
\eqref{eq:power-law-dist}); this represents the one-dimensional
``marginalized'' distribution
\begin{equation}
  \label{eq:alpha-pdf}
  p(\alpha|d) = \int d\Mmin\, d\Mmax\, p(\{\Mmin, \Mmax, \alpha\}|d),
\end{equation}
and similarly for $\Mmin$ and $\Mmax$.

\begin{figure}
  \begin{center}
    \plotone{plots/power-law}
  \end{center}
  \caption{\label{fig:power-law} Histograms of the marginalized
    distribution for the three parameters $\Mmin$, $\Mmax$, and
    $\alpha$ from the power-law model.  The marginalized distribution
    for $\alpha$ is broad, with $-11.8 < \alpha < 6.8$ enclosing 90\%
    of the probability.  We have $p(\alpha < 0) = 0.6$; the median
    value is $\alpha = -3.35$.  The broad distribution for $\alpha$
    (and the other parameters) is due to correlations between the
    parameters discussed in the main text; see Figure
    \ref{fig:power-law-2D}.}
\end{figure}

The marginalized distribution for $\alpha$ is broad, with
\begin{equation}
  -11.8 < \alpha < 6.8
\end{equation}
enclosing 90\% of the probability (excluding 5\% on each side).  We
have $p(\alpha < 0) = 0.6$.  The median value is $\alpha = -3.35$.
The broadness of the marginalized distribution for $\alpha$ comes from
the need to match the relatively narrow range in mass of the
lower-mass systems.  When $\alpha$ is negative, the resulting mass
distribution slopes down; $\Mmin$ is constrained to be near the lowest
mass of the observed black holes, while $\Mmax$ is essentially
irrelevant.  Conversely, when $\alpha$ is positive and the mass
distribution slopes up, $\Mmax$ must be close to the largest mass
observed, while $\Mmin$ is essentially irrelevant.  Figure
\ref{fig:power-law-2D} illustrates this effect, showing the
correlations between $\alpha$ and $\Mmin$ and $\alpha$ and $\Mmax$.
When we include the higher-mass systems in the analysis, the long tail
will eliminate this effect by bringing both $\Mmin$ and $\Mmax$ into
play for all values of $\alpha$.

\begin{figure}
  \begin{center}
    \plotone{plots/power-law-2D}
  \end{center}
  \caption{\label{fig:power-law-2D} MCMC samples in the $\Mmin,
    \alpha$ (top) and $\Mmax, \alpha$ (bottom) planes for the
    power-law model discussed in Section \ref{sec:power-law}.  The
    correlations between $\alpha$ and the power law bounds discussed
    in the text are apparent: when $\alpha$ is positive, the mass
    distribution slopes upward and $\Mmax$ is constrained to be near
    the maximum observed mass while $\Mmin$ is unconstrained.  When
    $\alpha$ is negative, the mass distribution slopes down and
    $\Mmin$ is constrained to be near the lowest mass observed, while
    $\Mmax$ is unconstrained. }
\end{figure}

\subsubsection{Decaying Exponential}

Figure \ref{fig:exp-marginal} displays the marginalized posterior
distribution for the scale mass of the exponential, $M_0$, and the
cutoff mass, $\Mmin$ (see Equation \ref{eq:exp-def}).  The median
scale mass is $M_0 = 1.55$, and $0.78 \leq M_0 \leq 3.25$ with 90\%
confidence.  This model was one of those considered by
\citet{Ozel2010}, whose results ($M_0 \sim 1.5$ and $\Mmin \sim 6.5$)
are broadly consistent with ours.  Figure \ref{fig:exp-2D} displays
the MCMC samples in the $\Mmin$, $M_0$ plane for this model.  There is
a small correlation between smaller $\Mmin$ and larger $M_0$, which is
driven by the need to widen the distribution to encompass the peak of
the mass measurements in Figure \ref{fig:low-masses} when the minimum
mass is smaller.

\begin{figure}
  \begin{center}
    \plotone{plots/exp-cutoff}
  \end{center}
  \caption{\label{fig:exp-marginal} The distribution of scale masses,
    $M_0$ (dashed histogram), and minimum masses, $\Mmin$ (solid
    histogram), both measured in units of a solar mass for the
    exponential underlying mass distribution defined in Equation
    \eqref{eq:exp-def}.  The median scale mass is $M_0 = 1.55$, and
    $0.78 \leq M_0 \leq 3.25$ with 90\% confidence.}
\end{figure}

\begin{figure}
  \begin{center}
    \plotone{plots/exp-cutoff-2d}
  \end{center}
  \caption{\label{fig:exp-2D} The MCMC samples in the $\Mmin$, $M_0$
    plane for the decaying exponential underlying mass distribution
    model.  The slight correlation between smaller $\Mmin$ and larger
    $M_0$ is driven by the need to widen the mass distribution to
    encompass the peak of the measurements in Figure
    \ref{fig:low-masses} when the minimum mass decreases.}
\end{figure}

\subsubsection{Gaussian}

Figure \ref{fig:gaussian} shows the resulting marginalized
distributions for the parameters $\mu$ and $\sigma$.  We constrain the
peak of the Gaussian between $6.63 \leq \mu \leq 8.08$ with 90\%
confidence.  This model also appeared in \citet{Ozel2010}; they found
$\mu \sim 7.8$ and $\sigma \sim 1.2$, consistent with our results
here.

\begin{figure}
  \begin{center}
    \plotone{plots/gaussian}
  \end{center}
  \caption{\label{fig:gaussian} Marginalized posterior distributions
    for the mean, $\mu$ (solid histogram), and standard deviation,
    $\sigma$ (dashed histogram), both in solar masses for the Gaussian
    underlying mass distribution defined in Equation
    \eqref{eq:gaussian-def}.  The peak of the Gaussian, $\mu$, is
    constrained in $6.63 \leq \mu \leq 8.08$ with 90\% confidence.}
\end{figure}

\subsubsection{Two Gaussian}

Figure \ref{fig:two-gaussian} shows the marginalized distributions for
the two-Gaussian model parameters from our MCMC runs.  We find $\alpha
> 0.8$ with 62\% probability, clearly favoring the first Gaussian.
The distributions for $\mu_1$ and $\sigma_1$ are similar to those of
the single Gaussian displayed in Figure \ref{fig:gaussian}, indicating
that the first Gaussian is centered around the peaks of the lower-mass
distributions.  The second Gaussian's parameter distributions are much
broader.  The second Gaussian appears to be sampling the tail of the
mass samples.  In spite of the ability to match a more complicated
distribution, we find that this model is strongly disfavored relative
to the single-Gaussian model for this dataset:
$p(\textnormal{Gaussian}|d) / p(\textnormal{Two Gaussian}|d) \simeq
4.7$ (see Sections \ref{sec:model-selection} and
\ref{sec:lower-mass-model-selection} for discussion).

\begin{figure}
  \begin{center}
    \plotone{plots/two-gaussian}
  \end{center}
  \caption{\label{fig:two-gaussian} The marginal distributions for the
    five parameters of the two-Gaussian model.  The top panel is
    $\mu_1$ (solid histogram) and $\sigma_1$ (dashed histogram), the
    middle panel is $\mu_2$ (solid histogram) and $\sigma_2$ (dashed
    histogram), and the bottom panel is $\alpha$. We have $\alpha >
    0.8$ with 62\% probability, favoring the first of the two
    Gaussians.  The distributions for $\mu_1$ and $\sigma_1$ are
    similar to those of the single Gaussian model displayed in Figure
    \ref{fig:gaussian}; the second Gaussian's parameter distributions
    are much broader (recall that we constrain $\mu_2 > \mu_1$).  The
    second Gaussian is attempting to fit the tail of the mass samples.
    The extra degrees of freedom in the distribution from the second
    Gaussian do not provide enough extra fitting power to compensate
    for the increase in parmeter space, however: the two-Gaussian
    model is disfavored relative to the single Gaussian by a factor of
    $4.7$ on this dataset (see Sections \ref{sec:model-selection} and
    \ref{sec:lower-mass-model-selection} for discussion).}
\end{figure}

\subsubsection{Log Normal}

The marginal distributions for $\langle M \rangle$ and $\sigma_M$
appear in Figure \ref{fig:log-normal}.  The distributions are similar
to those for $\mu$ and $\sigma$ from the Gaussian model in Section
\ref{sec:gaussian}.

\begin{figure}
  \begin{center}
    \plotone{plots/log-normal}
  \end{center}
  \caption{\label{fig:log-normal} Marginalized distributions of the
    mean mass, $\langle M \rangle$ (solid histogram), and standard
    deviation of the mass, $\sigma_M$ (dashed histogram), for the
    log-normal model in Section \ref{sec:log-normal}.  The
    distributions are similar to the distributions of $\mu$ and
    $\sigma$ in the Gaussian model of Section \ref{sec:gaussian}.}
\end{figure}

\subsubsection{Histogram Models}

The median values of the histogram mass distributions that result from
the MCMC samples of the posterior distribution for the $w_i$
parameters for one-, two-, three-, four-, and five-bin histogram
models are shown in Figure \ref{fig:dists}.  Table
\ref{tab:low-mass-non-parametric} gives quantiles of the marginalized
bin boundary distributions for the histogram models.

\begin{table}
  \begin{center}
    \begin{tabular}{|c|c|c|c|c|c|c|}
      \hline
      Bins & Boundary & 5\% & 15\% & 50\% & 85\% & 95\% \\
      \hline \hline
      1 & $w_0$ & 3.94488 & 4.55603 & 5.43333 & 6.02557 & 6.29749 \\
      \hline
        & $w_1$ & 8.50844 & 8.69262 & 9.11784 & 9.83477 & 10.5128 \\
      \hline \hline
      2 & $w_0$ & 3.3426 & 4.2047 & 5.39132 & 6.18413 & 6.47553 \\
      \hline
        & $w_1$ & 6.41972 & 6.72605 & 7.43421 & 8.2489 & 8.52885 \\
      \hline
        & $w_2$ & 8.46161 & 8.65077 & 9.12694 & 10.1113 & 11.2595 \\
      \hline \hline
      3 & $w_0$ & 2.18176 & 3.54345 & 5.16094 & 6.16473 & 6.44697 \\
      \hline
        & $w_1$ & 5.68876 & 6.14223 & 6.68829 & 7.38725 & 8.04235 \\
      \hline
        & $w_2$ & 6.8297 & 7.22718 & 8.1451 & 8.7512 & 9.27296 \\
      \hline
        & $w_3$ & 8.44307 & 8.67362 & 9.25718 & 12.1688 & 21.92 \\
      \hline \hline
      4 & $w_0$ & 1.32131 & 2.7934 & 4.66156 & 5.78459 & 6.17946 \\
      \hline
        & $w_1$ & 5.20112 & 5.77331 & 6.42501 & 6.98427 & 7.44584 \\
      \hline
        & $w_2$ & 6.41805 & 6.73535 & 7.43826 & 8.32958 & 8.64212 \\
      \hline
        & $w_3$ & 7.40302 & 7.95608 & 8.58976 & 9.33897 & 10.3992 \\
      \hline
        & $w_4$ & 8.56724 & 8.8059 & 10.2451 & 24.3573 & 34.2423 \\
      \hline \hline
      5 & $w_0$ & 0.9392 & 2.28789 & 4.33389 & 5.7012 & 6.21166 \\
      \hline
        & $w_1$ & 4.69778 & 5.44302 & 6.26575 & 6.76407 & 7.14427 \\
      \hline
        & $w_2$ & 6.1388 & 6.47155 & 7.00606 & 7.97325 & 8.38259 \\
      \hline
        & $w_3$ & 6.82058 & 7.28677 & 8.22514 & 8.81555 & 9.41012 \\
      \hline
        & $w_4$ & 8.02335 & 8.36993 & 8.94879 & 11.3206 & 17.3349 \\
      \hline
        & $w_5$ & 8.7112 & 9.25208 & 16.2059 & 31.897 & 37.2738 \\
      \hline
    \end{tabular}
  \end{center}
  \caption{\label{tab:low-mass-non-parametric} The 5\%, 15\%, 50\%
    (median), 85\%, and 95\% quantiles for the bin boundaries in the
    one- through five-bin histogram models discussed in Section
    \ref{sec:non-parametric-models}.}
\end{table}

As the number of bins increases, the models are better able to capture
features of the mass distribution, but we find that the one-bin
histogram is the most probable of the histogram models for the
low-mass data (see Section \ref{sec:lower-mass-model-selection} for
discussion).  This occurs because the extra fitting power does not
sufficiently improve the fit to compensate for the vastly larger
parameter space of the models with more bins.

\subsubsection{Lower-Mass Model Selection}
\label{sec:lower-mass-model-selection}
We have performed a suite of 500 independent reversible-jump MCMCs
jumping between all the models (both parametric and non-parametric) in
Section \ref{sec:models} using the single-model MCMC samples to
construct an efficient jump proposal for each model as described above
(see Appendix \ref{sec:reversible-jump-mcmc}).  The numbers of counts in
each model are consistent across the MCMCs in the suite; Figure
\ref{fig:rj} displays the average probability for each model across
the suite, along with the 1-$\sigma$ errors on the average inferred
from the standard deviation of the model counts across the suite.
Table \ref{tab:rj} gives the numerical values of the average
probability for each model across the suite of MCMCs.

\begin{figure}
  \begin{center}
    \plotone{plots/rj}
  \end{center}
  \caption{\label{fig:rj} The relative probability of the models
    discussed in Section \ref{sec:models} as computed using the
    reversible-jump MCMC with the efficient jump proposal algorithm
    described in Section \ref{sec:reversible-jump-mcmc}.  In
    increasing order along the $x$-axis, the models are the power-law
    of Section \ref{sec:power-law} (PL), the decaying exponential of
    Section \ref{sec:exponential} (E), the single Gaussian of Section
    \ref{sec:gaussian} (G), the double Gaussian of Section
    \ref{sec:gaussian} (TG), and the one-, two-, three-, four-, and
    five-bin histogram models of Section
    \ref{sec:non-parametric-models} (H1, H2, H3, H4, H5,
    respectively).  The average of 500 independent reversible-jump
    MCMCs is plotted, along with the 1-$\sigma$ error on the average
    inferred from the standard deviation of the probability from the
    individual MCMCs.  As discussed in the text, the power-law and
    Gaussian models are the most favored.}
\end{figure}

The most favored model is the power law from Section
\ref{sec:power-law}, followed by the Gaussian model from Section
\ref{sec:gaussian}.  Interestingly, the theoretical curve from
\citet{Fryer2001} (the exponential model of Section
\ref{sec:exponential}) places fourth in the ranking of evidence.

\begin{table}
  \begin{center}
    \begin{tabular}{|l|r|}
      \hline
      Model & Relative Evidence \\
      \hline \hline
      Power Law (Section \ref{sec:power-law}) & 0.331488 \\
      \hline
      Gaussian (Section \ref{sec:gaussian}) & 0.288129 \\ 
      \hline
      Log Normal (Section \ref{sec:log-normal}) & 0.138435 \\
      \hline
      Exponential (Section \ref{sec:exponential}) & 0.0916218 \\
      \hline
      Two Gaussian (Section \ref{sec:gaussian}) & 0.0662577 \\
      \hline
      Histogram (1 Bin, Section \ref{sec:non-parametric-models}) &
      0.0641941 \\
      \hline
      Histogram (2 Bin, Section \ref{sec:non-parametric-models}) &
      0.015184 \\
      \hline 
      Histogram (3 Bin, Section \ref{sec:non-parametric-models}) &
      0.00332933 \\
      \hline
      Histogram (4 Bin, Section \ref{sec:non-parametric-models}) &
      0.000999976 \\
      \hline 
        Histogram (5 Bin, Section \ref{sec:non-parametric-models}) &
      0.0003614  \\
      \hline      
    \end{tabular}
  \end{center}
  \caption{\label{tab:rj} Relative probabilities of the various models
    from Section \ref{sec:models} implied by the lower-mass data.  These probabilities have been 
    computed from reversible-jump
    MCMC samples using the efficient jump proposal algorithm in Appendix \ref{sec:reversible-jump-mcmc}.}
\end{table}

Though the model probabilities presented in this section have small
statistical error, they are subject to large ``systematic'' error.
The source of this error is both the particular choice of model prior
(uniform across models) and the choice of priors on the parameters
within each model used for this work.  For example, the
theoretically-preferred exponential model (Section
\ref{sec:exponential}) is only a factor of $\sim 3$ away from the
power law model (Section \ref{sec:power-law}), which does not have
theoretical support.  Is such support worth a factor of three in the
model prior?  Alternately, we may say we know (in advance of any mass
measurements) that black holes must exist with mass $\lesssim
10\Msun$; then we could, for example, impose a prior on the minimum
mass in the exponential model ($\Mmin$) that is uniform between $0$
and $10 \Msun$, which would reduce the prior volume available for the
model by a factor of 4 without significantly reducing the posterior
support for the model.  This has the same effect as increasing the
model prior by a factor of 4, which would move this model from fourth
to first place.  Of course, we would then have to modify the prior
support for the other models to take into account the restriction that
there must be black holes with $M \lesssim 10\Msun$....
\citet{Linder2008} discuss these issues in the context of cosmological
model selection, concluding with a similar warning against
over-reliance on model selection probabilities.

Nevertheless, we believe that our model comparison is reasonably fair
(see the discussion of priors in Section \ref{sec:priors}).  It seems
safe to conclude that ``single-peaked'' models (the power-law and
Gaussian) are preferred over ``extended'' models (the exponential or
log-normal), or those with ``structure'' (the many-bin histograms or
two-Gaussian model).  Previous studies have also supported the
``single, narrow peak'' mass distribution \citep{Bailyn1998,Ozel2010}.
In this light, poor performance of the single-bin histogram is
surprising.

\subsection{Complete Sample}
\label{sec:higher-mass}

This section repeats the analysis of the models from Section
\ref{sec:models}, but including the higher-mass, wind-fed systems from
Table \ref{tab:sources} (see also Figure \ref{fig:high-masses}) in the
sample.  Figure \ref{fig:high-mass-dists} displays bounds on the value
of the underlying mass distribution for the various models in Section
\ref{sec:models} applied to this data set; compare to Figure
\ref{fig:dists}.  The inclusion of the higher-mass, wind-fed systems
tends to widen the distribution toward the high-mass end and, in
models that allow it, produce a second, higher-mass peak in addition
to the one in Figure \ref{fig:dists}.

\begin{figure}
  \begin{center}
    \plotone{plots/dist-high}
  \end{center}
  \caption{\label{fig:high-mass-dists} The median (solid line), 10\%
    (lower dashed line), and 90\% (upper dashed line) values of the
    black hole mass distribution, $p(M|\theta)$, at various masses
    implied by the posterior $p(\theta|d)$ for the models discussed in
    Sections \ref{sec:parametric-models} and
    \ref{sec:non-parametric-models}.  These distributions use the
    complete sample of 20 observations in Table \ref{tab:sources},
    including the higher-mass, wind-fed systems.  Note that these
    ``distributions of distributions'' are not necessarily normalized,
    and need not be ``shaped'' like the underlying model
    distributions.  Compare to Figure \ref{fig:dists}, which includes
    only the low-mass systems in the analysis.  Including the
    higher-mass systems tends to widen the distribution toward the
    high-mass end and, in models that allow it, produce a second,
    higher-mass peak in addition to the one in Figure
    \ref{fig:dists}. }
\end{figure}

\subsubsection{Power Law}

Figure \ref{fig:power-law-high} presents the marginalized
distribution for the three power-law parameters $\Mmin$, $\Mmax$, and
$\alpha$ (Section \ref{sec:power-law}) from an analysis including the
higher-mass systems.  The distribution for $\Mmax$ is quite broad
because the best fit power laws slope downward ($\alpha < 0$), making
this parameter less relevant.  The range $-5.05 \leq \alpha \leq
-1.77$ encloses 90\% of the probability; the median value of $\alpha$
is -3.23.  The presence of the higher-mass samples in the analysis
produces a distinctive tail, eliminating the correlations discussed in
Section \ref{sec:power-law} and displayed in Figure
\ref{fig:power-law-2D} for the lower-mass subset of the observations.

\begin{figure}
  \begin{center}
    \plotone{plots/power-law-high}
  \end{center}
  \caption{\label{fig:power-law-high} Histograms of the marginalized
    distribution for the three parameters $\Mmin$, $\Mmax$, and
    $\alpha$ from the power-law model including the higher-mass
    samples in the MCMC.  The distribution for $\Mmax$ is quite broad
    because the best fit power laws slope downward ($\alpha < 0$),
    making this parameter less relevant.  The range $-5.05 \leq \alpha
    \leq -1.77$ encloses 90\% of the probability; the median value of
    $\alpha$ is -3.23.  The presence of the higher-mass samples in the
    analysis produces a distinctive tail, eliminating the correlations
    discussed in Section \ref{sec:power-law} and displayed in Figure
    \ref{fig:power-law-2D} for the lower-mass subset of the
    observations. }
\end{figure}

\subsubsection{Decaying Exponential}

Figure \ref{fig:exp-cutoff-high} displays the marginalized
distributions for the exponential parameters $\Mmin$ and $M_0$
(Section \ref{sec:exponential}) from an analysis including the
higher-mass systems.  The distribution for the scale mass, $M_0$, has
moved to higher masses relative to Figure \ref{fig:exp-marginal} to
fit the tail of the mass distribution; the distribution for $\Mmin$ is
less affected, though it has broadened somewhat toward low masses.

\begin{figure}
  \begin{center}
    \plotone{plots/exp-cutoff-high}
  \end{center}
  \caption{\label{fig:exp-cutoff-high} The marginalized distributions
    for the exponential parameters $\Mmin$ and $M_0$ (Section
    \ref{sec:exponential}) from an analysis including the higher-mass
    systems.  The distribution for the scale mass, $M_0$, has moved to
    higher masses relative to Figure \ref{fig:exp-marginal} to fit the
    tail of the mass distribution; we now have $2.8292 \leq M_0 \leq
    7.9298$ with 90\% confidence, with median 4.7003.  The
    distribution for $\Mmin$ is less affected, though it has broadened
    somewhat toward low masses.}
\end{figure}

\subsubsection{Gaussian}

Figure \ref{fig:gaussian-high} displays the marginalized distributions
for the Gaussian parameters (Section \ref{sec:gaussian}) when the
higher-mass objects are included in the mass distribution.  The mean
mass, $\mu$, and the mass standard deviation, $\sigma$, are both
increased relative to Figure \ref{fig:gaussian} to account for the
broader distribution and higher-mass tail.

\begin{figure}
  \begin{center}
    \plotone{plots/gaussian-high}
  \end{center}
  \caption{\label{fig:gaussian-high} The marginalized distributions
    for the Gaussian parameters when the higher-mass objects are
    included in the mass distribution.  The mean mass, $\mu$, and the
    mass standard deviation, $\sigma$, are both increased relative to
    Figure \ref{fig:gaussian} to account for the broader distribution
    and higher-mass tail.  The peak of the underlying mass
    distribution lies in the range $7.8660 \leq \mu \leq 10.9836$ with
    90\% confidence; the median value is 9.2012.}
\end{figure}

\subsubsection{Two Gaussian}

The analysis of the two-Gaussian model shows the largest change when
the higher-mass samples are included.  Figure
\ref{fig:two-gaussian-high} shows the marginalized distributions for
the two-Gaussian parameters (Section \ref{sec:gaussian}) when the
higher-mass samples are included in the analysis.  In stark contrast
to Figure \ref{fig:two-gaussian}, there are two well-defined,
separated peaks; the lower-mass peak reproduces the results from the
low-mass samples, while the higher-mass peak ($13.5534 \leq \mu_2 \leq
27.9481$ with 90\% confidence; median 20.3839) matches the new
higher-mass samples.  The peak in $\alpha$ near 0.8 is consistent with
approximately 4/5 the total probability being concentrated in the 15
low-mass samples.

\begin{figure}
  \begin{center}
    \plotone{plots/two-gaussian-high}
  \end{center}
  \caption{\label{fig:two-gaussian-high} The marginalized
    distributions for the two-Gaussian parameters (Section
    \ref{sec:gaussian}) when the higher-mass samples are included in
    the analysis.  In stark contrast to Figure \ref{fig:two-gaussian},
    there are two well-defined, separated peaks; the lower-mass peak
    reproduces the results from the low-mass samples, while the
    higher-mass peak ($13.5534 \leq \mu_2 \leq 27.9481$ with 90\%
    confidence; median 20.3839) matches the new higher-mass samples.
    The peak in $\alpha$ near 0.8 is consistent with approximately 15
    out of 20 samples belonging to the low-mass peak.}
\end{figure}

\subsubsection{Log Normal}

The marginalized distributions for the log-normal parameters (Section
\ref{sec:log-normal}) when the higher-mass samples are included in the
analysis are displayed in Figure \ref{fig:log-normal-high}.  The
changes when the higher-mass samples are included (compare to Figure
\ref{fig:log-normal}) are similar to the changes in the Gaussian
distribution: the mean mass moves to higher masses, and the
distribution broadens.  Because the log-normal distribution is
inherently asymmetric, with a high-mass tail, it does not need to
widen as much as the Gaussian distribution did.

\begin{figure}
  \begin{center}
    \plotone{plots/log-normal-high}
  \end{center}
  \caption{\label{fig:log-normal-high} The marginalized distributions
    for the log-normal parameters (Section \ref{sec:log-normal}) when
    the higher-mass samples are included in the analysis.  The changes
    when the higher-mass samples are included (compare to Figure
    \ref{fig:log-normal}) are similar to the changes in the Gaussian
    distribution: the mean mass moves to higher masses, and the
    distribution broadens.}
\end{figure}

The confidence limits on the parameters for the parametric models of
the underlying mass distribution are displayed in Table
\ref{tab:high-mass-parametric} (compare to Table
\ref{tab:low-mass-parametric}).

\begin{table}
  \begin{center}
    \begin{tabular}{|l|c|c|c|c|c|c|}
      \hline
      Model & Parameter & 5\% & 15\% & 50\% & 85\% & 95\% \\
      \hline \hline
      Power Law (Equation \eqref{eq:power-law-dist}) & $\Mmin$ & 
      4.87141 & 5.29031 & 5.85019 & 6.26118 & 6.45674 \\
      \hline
      & $\Mmax$ & 19.1097 & 23.4242 & 31.5726 & 37.7519 & 39.3369 \\
      \hline
      & $\alpha$ & -5.04879 & -4.30368 & -3.23404 & -2.31365 & -1.77137 \\
      \hline \hline
      Exponential (Equation \eqref{eq:exp-def})& $\Mmin$ & 
      4.0865 & 4.60236 & 5.32683 & 5.94097 & 6.22952 \\
      \hline
       & $M_0$ & 2.82924 & 3.41139 & 4.70034 & 6.52214 & 7.92979 \\
      \hline \hline
      Gaussian (Equation \eqref{eq:gaussian-def}) & $\mu$ & 
      7.86599 & 8.33118 & 9.20116 & 10.2493 & 10.9836 \\
      \hline
      & $\sigma$ & 2.23643 & 2.58899 & 3.33545 & 4.17886 & 4.67881 \\
      \hline \hline
      Two Gaussian (Equation \eqref{eq:two-gaussian-def}) & $\mu_1$ & 
      6.741 & 7.02724 & 7.48174 & 8.0139 & 8.46626 \\
      \hline
      & $\mu_2$ & 13.5534 & 16.202 & 20.3839 & 24.9259 & 27.9481 \\
      \hline
      & $\sigma_1$ & 0.742824 & 0.913941 & 1.31244 & 1.94862 & 2.50238 \\
      \hline
      & $\sigma_2$ & 0.511159 & 1.5025 & 4.39824 & 7.04612 & 8.25905 \\
      \hline
      & $\alpha$ & 0.575692 & 0.670978 & 0.798227 & 0.891522 & 0.932143 \\
      \hline \hline
      Log Normal (Equation \eqref{eq:log-normal-def}) & $\langle M \rangle$ & 
      8.00086 & 8.51192 & 9.6264 & 11.1851 & 12.3986 \\
      \hline
      & $\sigma_M$ & 2.19262 & 2.8137 & 4.16742 & 6.25101 & 8.11839 \\
      \hline
    \end{tabular}
  \end{center}
  \caption{\label{tab:high-mass-parametric} Quantiles of the
    marginalized distribution for each of the parameters in the models
    discussed in Section \ref{sec:parametric-models} when the
    higher-mass samples are included in the analysis (compare to Table
    \ref{tab:low-mass-parametric}).  We indicate the 5\%, 15\%, 50\% 
    (median), 85\%, and 95\% quantiles.}
\end{table}

\subsubsection{Histogram Models}

The non-parametric (histogram; see Section
\ref{sec:non-parametric-models}) models also show evidence of a long
tail from the inclusion of the higher-mass samples.  Table
\ref{tab:high-mass-non-parametric} displays confidence limits on the
histogram parameters for the analysis including the higher-mass
systems; compare to Table \ref{tab:low-mass-non-parametric}.

\begin{table}
  \begin{center}
    \begin{tabular}{|c|c|c|c|c|c|c|}
      \hline
      Bins & Boundary & 5\% & 15\% & 50\% & 85\% & 95\% \\
      \hline \hline
      \hline
      1 & $w_0$ & 2.22294 & 3.12695 & 4.2456 & 5.15132 & 5.58265 \\
      \hline
      & $w_1$ & 15.93 & 16.2535 & 17.7836 & 20.5449 & 22.5836 \\
      \hline \hline
      2 & $w_0$ & 3.87202 & 4.49983 & 5.41234 & 6.08334 & 6.35933 \\
      \hline
      & $w_1$ & 7.22163 & 8.25079 & 8.93669 & 9.71551 & 10.4287 \\
      \hline
      & $w_2$ & 18.4762 & 19.9798 & 24.941 & 32.5972 & 36.8615 \\
      \hline \hline
      3 & $w_0$ & 3.39289 & 4.24509 & 5.41694 & 6.15087 & 6.42822 \\
      \hline
      & $w_1$ & 6.41849 & 6.71984 & 7.47263 & 8.2942 & 8.61785 \\
      \hline
      & $w_2$ & 8.41449 & 8.64664 & 9.17056 & 10.4075 & 12.2718 \\
      \hline
      & $w_3$ & 18.5705 & 21.0481 & 27.1494 & 34.7753 & 38.0652 \\
      \hline \hline
      4 & $w_0$ & 2.42094 & 3.69875 & 5.2596 & 6.25449 & 6.54316 \\
      \hline
      & $w_1$ & 5.83725 & 6.2836 & 6.84987 & 7.8033 & 8.27706 \\
      \hline
      & $w_2$ & 6.94919 & 7.43628 & 8.38531 & 9.13401 & 9.91845 \\
      \hline
      & $w_3$ & 8.50371 & 8.75188 & 9.86694 & 17.1848 & 22.1086 \\
      \hline
      & $w_4$ & 18.5823 & 21.4628 & 28.367 & 35.8118 & 38.5278 \\      
      \hline \hline
      5 & $w_0$ & 1.73691 & 3.19184 & 4.89769 & 5.9547 & 6.35522 \\
      \hline
      & $w_1$ & 5.46124 & 5.95881 & 6.59431 & 7.26795 & 7.91821 \\
      \hline
      & $w_2$ & 6.63468 & 6.9804 & 7.93239 & 8.60918 & 9.06926 \\
      \hline
      & $w_3$ & 7.89654 & 8.35634 & 8.91766 & 10.6568 & 13.9644 \\
      \hline
      & $w_4$ & 8.74064 & 9.42672 & 15.8004 & 22.7101 & 27.6399 \\
      \hline
      & $w_5$ & 20.0202 & 22.9065 & 29.6307 & 36.6606 & 38.8573 \\
      \hline
    \end{tabular}
  \end{center}
  \caption{\label{tab:high-mass-non-parametric} The 5\%, 15\%, 50\%
    (median), 85\%, and 95\% quantiles for the bin boundaries in the
    one- through five-bin histogram models discussed in Section
    \ref{sec:non-parametric-models} in an 
    analysis including the higher-mass, wind-fed systems.  
    The tails evident in Figure \ref{fig:high-mass-dists} are apparent
    here as well; compare to Table \ref{tab:low-mass-non-parametric}.}
\end{table}

\subsubsection{High-Mass Model Selection}
\label{sec:high-mass-model-selection}

Repeating the model selection analysis discussed in Section
\ref{sec:lower-mass-model-selection} for the sample including the
higher-mass systems, we find that the model probabilities have changed
with the inclusion of the extra five systems.  As before, we assume
for this analysis that the model priors are equal.

Reversible jump MCMC calculations of the model probabilities are
displayed in Figure \ref{fig:high-rj-evidence}; compare Figure
\ref{fig:rj}.  The exponential model is the most favored model for the
complete sample, with the two-Gaussian model the second-most favored.
The ranking of models differs significantly from the lower-mass
samples.  The improvement of the exponential model relative to the
lower-mass analysis is encouraging for theoretical calculations that
attempt to model the entire population of X-ray binaries with this
mass model.  Note also that the increasing structure of the mass
distribution favors histogram models with three bins over those with
fewer bins.

\begin{figure}
  \begin{center}
    \plotone{plots/rj-high}
  \end{center}
  \caption{\label{fig:high-rj-evidence} The relative probability of
    the models discussed in Section \ref{sec:models} as computed using
    the reversible-jump MCMC with the efficient jump proposal
    algorithm described in Section \ref{sec:reversible-jump-mcmc},
    applied to all 20 systems in Table \ref{tab:sources} (i.e.\
    including the higher-mass systems).  In increasing order along the
    $x$-axis, the models are the power-law of Section
    \ref{sec:power-law} (PL), the decaying exponential of Section
    \ref{sec:exponential} (E), the single Gaussian of Section
    \ref{sec:gaussian} (G), the double Gaussian of Section
    \ref{sec:gaussian} (TG), and the one-, two-, three-, four-, and
    five-bin histogram models of Section
    \ref{sec:non-parametric-models} (H1, H2, H3, H4, H5,
    respectively).  The average of 500 independent reversible-jump
    MCMCs is plotted, along with the 1-$\sigma$ error on the average
    inferred from the standard deviation of the probability from the
    individual MCMCs.  Compare to Figure \ref{fig:rj}.}
\end{figure}


\section{The Minimum Mass of the Black Hole Mass Distribution}
\label{sec:minimum-mass}

It is interesting to use our models for the underlying mass
distribution of X-ray binary black holes to try to place constraints
on the minimum black hole mass.  \citet{Bailyn1998} addressed this
question in the context of a ``mass gap'' between the most massive
neutron stars and the least massive black holes.  The more recent
study of \citet{Ozel2010} also looked for a mass gap using a subset of
the models and systems presented here.  Both works found that the
minimum black hole mass is significantly above the maximum neutron
star mass.

The distribution of the minimum black hole mass from the analysis of
the lower-mass samples is displayed in Figure \ref{fig:min-mass}.  For
all distributions, the minimum black hole mass is defined as the 1\%
mass quantile (i.e.\ the mass lying below 99\% of the mass
distribution).  (A cutoff quantile is necessary in the case of those
distributions that do not have a hard cutoff mass; even for those that
do, like the power-law model, it can be useful to define a ``soft''
cutoff in the event that the lower mass hard cutoff becomes an
irrelevant parameter as discussed in Section \ref{sec:power-law}.)

\begin{figure}
  \begin{center}
    \plotone{plots/mmin}
  \end{center}
  \caption{\label{fig:min-mass} The distributions for the minimum
    black hole mass calculated from the MCMC samples for the models in
    Section \ref{sec:models} applied to the lower-mass systems.  The
    minimum black hole mass is defined as the 1\% mass quantile.  For
    the most favored models, the power-law and Gaussian, the 90\%
    confidence limit on the minimum black hole mass is 4.3 $\Msun$ and
    2.9 $\Msun$, respectively.}
\end{figure}

We find that the best-fit model for the lower-mass systems (the
power-law) has a 90\% confidence limit on the minimum black hole mass
of 4.3 $\Msun$.  This is significantly above the maximum
theoretically-allowed neutron star mass, $\sim 3 \Msun$
\citep{Kalogera1996}\fixme{Vicky, do you have a more recent reference
  than this?}, so we conclude that the lower-mass systems show strong
evidence of a mass gap.

The distribution of minimum black hole masses for the analysis of the
complete sample (i.e.\ including the higher-mass systems) is shown in
Figure \ref{fig:high-min-mass}.  For the most favored model, the
exponential, the 90\% confidence limit on the minimum black hole mass
is 4.5 $\Msun$.  We therefore conclude that there is strong evidence
for a mass gap in the complete sample as well.

\begin{figure}
\begin{center}
    \plotone{plots/mmin-high}
  \end{center}
  \caption{\label{fig:high-min-mass} The distributions for the minimum
    black hole mass calculated from the MCMC samples for the models in
    Section \ref{sec:models} using the complete sample of systems.
    The minimum black hole mass is defined as the 1\% mass quantile.
    For the most favored model, the exponential, the 90\% confidence
    limit on the minimum black hole mass is 4.5 $\Msun$.}
\end{figure}

\section{Conclusion}

We have presented a Bayesian analysis of the mass distribution of
stellar-mass black holes in X-ray binary systems.  We considered
separately a sample of 15 lower-mass, Roche lobe-filling systems and a
sample of 20 systems containing the 15 lower-mass systems and five
higher-mass, wind-fed X-ray binaries.  We used MCMC methods to sample
the posterior distributions of the parameters for five parametric
models and five non-parametric (histogram) models for the mass
distribution that are implied by these data.  For both sets of
systems, we used reversible jump MCMCs (exploiting a new algorithm for
efficient jump proposals in such calculations) to perform model
selection on the suite of models.

For the lower-mass systems, we found the limits on model parameters in
Tables \ref{tab:low-mass-parametric} and
\ref{tab:low-mass-non-parametric}.  The relative model probabilites
from the model selection are given in Table \ref{tab:rj}.  The most
favored model for the lower-mass systems is a power law.  The
equivalent limits on the model parameters for the combined systems are
given in Tables \ref{tab:high-mass-parametric} and
\ref{tab:high-mass-non-parametric}.  Unlike the lower-mass systems,
the most favored model for the combined sample is the exponential
model.

We found strong evidence for a mass gap between the most massive
neutron stars, $\sim 3 \Msun$ \citep{Kalogera1996}, and the least
massive black holes.  For the lower-mass systems, the most favored,
power law model gives a black hole mass distribution whose 1\%
quantile lies above $4.3 \Msun$ with 90\% confidence.  For the
complete sample of systems, the most favored, exponential model gives
a black hole mass distribution whose 1\% quantile lise above $4.5
\Msun$ with 90\% confidence.  The existence of a mass gap was also
observed in \citet{Bailyn1998} (which used different models in a study
of 7 systems) and \citet{Ozel2010} (which did not contain a power law
model, and applied the exponential model to a sample of 16 lower-mass
systems, where it is strongly disfavored).  The existence of a mass
gap poses a challenge to current theoretical models of compact object
formation \citep{Fryer2001}, which predict a continuous mass
distribution from neutron star masses to stellar mass black hole
masses.

\acknowledgements

This work was supported by grant \fixme{XXX}.  IM acknowledges support
from the NSF AAPF under award AST-0901985.  Calculations for this work
were performed on the Northwestern Fugu cluster, which was partially
funded by NSF MRI grant \fixme{XXX}.  We thank Jonathan Gair for
helpful discussions.

\appendix

\section{Markov Chain Monte Carlo}

MCMC methods produce a Markov chain (or sequence) of parameter
samples, $\{ \vtheta_i \, | \, i = 1, \ldots \}$, such that a particular
parameter set, $\vtheta$, appears in the sequence with a frequency
equal to its probability according to a posterior, $p(\vtheta|d)$.  A
Markov chain has the property that the transition probability from one
element to the next, $p(\vtheta_i \to \vtheta_{i+1})$, depends only on
the value of $\vtheta_i$, not on any previous values in the chain.

One way to produce a sequence of MCMC samples is via the following
algorithm, first proposed by \citet{Metropolis1953} and used widely in
the physical sciences thereafter:
\begin{enumerate}
  \item Begin with the current sample, $\vtheta_i$.
  \item Propose a new sample, $\vtheta_p$, by drawing randomly from a
    ``jump proposal distribution'' with probability $Q(\vtheta_i \to
    \vtheta_p)$.  Note that $Q(\vtheta_i \to \vtheta_p)$ can depend on
    the current parameters, $\vtheta_i$, and any other ``constant''
    data, but cannot examine the history of the chain beyond the most
    recent point.  This is necessary to preserve the Markovian
    property of the chain.
  \item Compute the ``acceptance'' probability,
    \begin{equation}
      \label{eq:paccept}
      p_{\textnormal{accept}} \equiv
      \frac{p(\vtheta_p|d)}{p(\vtheta_i|d)} \frac{Q(\vtheta_p \to
        \vtheta_i)}{Q(\vtheta_i \to \vtheta_p)}
    \end{equation}
  \item With probability $\min(1,p_{\textnormal{accept}})$ ``accept''
    the propsed $\vtheta_p$, setting $\vtheta_{i+1} = \vtheta_p$;
    otherwise set $\vtheta_{i+1} = \vtheta_i$.
\end{enumerate}
This algorithm is more likely to accept a proposed jump when it
increases the posterior (the first factor in Equation
\eqref{eq:paccept}) and when it is to a location in parameter space
from which it is easy to return (the second factor in Equation
\eqref{eq:paccept}); the combination of these influences in Equation
\eqref{eq:paccept} ensures that the equilibrium distribution of the
chain is $p(\vtheta|d)$.  As $i \to \infty$ the samples $\vtheta_i$
are distributed according to $p(\vtheta|d)$.  

In practice the number of samples required before the chain
appropriately samples $p(\vtheta|d)$ depends strongly on the jump
proposal distribution; proposal distributions that often propose jumps
toward or within regions of large $p(\vtheta|d)$ can be very
efficient, while poor proposal distributions can require prohibitively
large numbers of samples before convergence.  

There is no foolproof test for the convergence of a chain.  In this
work, we test the convergence of our chains by comparing the
statistics calculated from the entire chain to statistics calculated
from only the first half of the chain; when the chain has converged,
the two calculations agree.  This is a necessary, but not sufficient,
condition for convergence.

We begin the chain at an arbitrary point in parameter space; this is
equivalent to taking a finite section of an infinite chain that begins
with the chosen point.  Every point in parameter space occurs in an
infinite chain, and no section of the chain is better than any other,
so a sufficiently long, but finite, section of the infinite chain
chosen in this manner can be representative of the statistics of the
chain as a whole.  However, because consecutive samples in a chain are
correlated with each other, the beginning of our finite chain has a
``memory'' of the starting point; we discard enough points at the
beginning of the finite chain that we can be confident it does not
retain a memory of the arbitrary starting point.  The points discarded
in this way are commonly called ``burn-in'' points.

\section{Reversible-Jump MCMC}
\label{sec:reversible-jump-mcmc}

Consider the problem of model selection among a set of models, and the
``super-model'' that encompasses all the models under consideration.
The parameter space of the super-model consists of a discrete
parameter that identifies the choice of model, $M_i$, and the
continuous parameters appropriate for this model, $\vtheta_i$.  We
denote a point in the super-model parameter space by $\{M_i,
\vtheta_i\}$; each such point is a statement that, e.g., ``the
underlying mass distribution for black holes in the galaxy is a
Gaussian, with parameters $\mu$ and $\sigma$,'' or ``the underlying
mass distribution for black holes in the galaxy is a triple-bin
histogram with parameters $w_1$, $w_2$, $w_3$, and $w_4$,'' or ....
To compare models, we are interested in the quantity (see Equation
\eqref{eq:model-posterior-def})
\begin{equation}
  p(M_i|d) = \int d\vtheta_i p(M_i, \vtheta_i|d).
\end{equation}
If we perform an MCMC in the super-model parameter space, then we
obtain a chain of samples $\{M_i, \vtheta_i \, | \, i = 1, \ldots\}$
distributed in parameter space with density $p(M_i,\vtheta_i|d)
d\vtheta_i$ and we can estimate the integral as
\begin{equation}
    p(M_i|d) = \int d\vtheta_i p(M_i, \vtheta_i|d) \approx \frac{N_i}{N},
\end{equation}
where $N_i$ is the number of samples in the chain lying in the
parameter space of model $M_i$ and $N$ is the total number of samples
in the chain.  The fraction of samples lying in the parameter space of
model $M_i$ gives the probability of that model relative to the other
models under consideration.

To perform the MCMC in the super-model parameter space, we must
propose jumps not only between points in a particular model's
parameter space, but also between the parameter spaces of different
models.  For this MCMC to be efficient, proposed jumps into a model
from another should favor regions with large posterior; when the
posterior is highly-peaked in a small region of parameter space,
proposed jumps outside this region are unlikely to be accepted, and
the reversible-jump MCMC samples will require a very long chain to
properly sample the ``super-model'' posterior.

We can exploit the information we have from single-model MCMCs to
generate efficient jump proposal distributions for our reversible jump
MCMC.  We would like to propose jumps that roughly follow the
distribution of samples in the single-model MCMCs.  We can do this by
assigning a neighborhood to each point in the sample using an
algorithm we will describe in the following paragraphs; the
neighborhoods are non-overlapping, completely cover the region of
parameter space with prior support, and contain only one point from
the MCMC samples.  To propose a jump into model $M_i$, we choose a
point uniformly from that single-model MCMC and then propose a jump
drawn uniformly from that point's neighborhood.  This is equivalent to
drawing from a piecewise-constant approximation to the single-model
posterior, where each neighborhood contributes a constant fraction,
$1/N_i$, to the cumulative jump probability.  In regions of high
density the neighborhoods are smaller, and the jump probability
density is correspondingly higher.  Because the neighborhoods cover
the entire region of prior support, it is possible for the proposal to
propose any point in parameter space with prior support (though points
in regions of low single-model posterior are of course unlikely to be
proposed).

To assign a neighborhood to each point in a single-model MCMC we use a
data structure called a kD-tree.  A kD-tree is a binary
space-partitioning tree.  To construct a kD-tree, we begin with the
set of points from a single-model MCMC and a box in parameter space
bounding the region of prior support (which must necessarily enclose
all the points).  The construction proceeds recursively: we choose a
dimension%
\footnote{Our algorithm chooses the dimension along which the
  numerical extent of the points is largest.  Other choices are
  possible; some algorithms cycle through the dimensions in order,
  while others choose a random dimension for each subdivision.  Our
  goal by picking the longest dimension is to produce neighborhoods
  that are ``square,'' at least in the chosen parametrization.} %
along which to divide the points, find the median point along that
dimension and its nearest neighbor, and divide the box at the midpoint
between these two points, producing two sub-boxes.  We then partition
the points into those to the left (i.e.\ smaller coordinate along the
given dimension) and right of the dividing line, and repeat this
procedure for each subset and the corresponding bounding box, until we
have only one point in each box.  An example of the neighborhoods that
result from a two-dimensional kD-tree constructed around a Gaussian
point distribution appears in Figure \ref{fig:kD-tree}.

\begin{figure}
  \begin{center}
    \plotone{plots/kdtree}
  \end{center}
  \caption{\label{fig:kD-tree} The neighborhoods constructed from a
    two-dimensional kD-tree built from a sample of points with a
    Gaussian density distribution.  Each line on the figure
    corresponds to a sub-dividing box boundary drawn between the
    median of a subset of the sample points and its nearest
    neighbor. The peak of the Gaussian lies in the center of the
    figure; here the point density is highest and the neighborhoods
    are smallest.  Near the edges the density is lower, and the
    neighborhoods correspondingly larger.  The tree adapts itself to
    the local density of points.  If these were single-model MCMC
    samples, the corresponding jump proposal would first select one of
    the boxes uniformly at random, and then choose a point uniformly
    within the box to propose.  Since there are many more boxes near
    the center (each box corresponds to one point), and these boxes
    are smaller, the proposal will tend to concentrate its points
    there, approximately tracking the distribution of single-model
    MCMC samples.}
\end{figure}

Construction of a kD-tree is an $\order{N\log N}$ operation, where $N$
is the number of points in the tree.  Median finding is $\order{n}$,
where $n$ is the number of points from which the median is to be
obtained.  At level $i$ in the tree, there are $2^i$ subsets of
points, each of length $\order{N/2^i}$, so the total cost of the $2^i$
median calculations is $\order{N}$ at each level.  There are
$\order{\log N}$ levels in the tree, yielding a total construction
cost for the tree of $\order{N \log N}$.  

To find the neighborhood of a point using the tree, we begin at the
root of the tree, and examine the two sub-boxes at the next level
down.  The point will be in one of them; following that branch, we
have again two sub-boxes, one of which contains the point; following
that branch....  Eventually, the search terminates at a leaf of the
tree, containing the point in question.  The box at the leaf defines
the neighborhood of the point in the jump proposal algorithm described
above.  The total cost for this operation is proportional to the
number of levels in the tree, which is $\order{\log N}$.

\bibliography{paper}

\end{document}