\documentclass[preprint]{aastex}

\usepackage{hyperref}
\usepackage{amsmath}

\newcommand{\fixme}[1]{[FIXME: #1]}
\newcommand{\Msun}{M_\odot}
\newcommand{\Mmin}{M_{\textnormal{min}}}
\newcommand{\Mmax}{M_{\textnormal{max}}}
\newcommand{\Nbin}{N_{\textnormal{bin}}}

\bibliographystyle{hapj}

\begin{document}

\title{The Mass Distribution of Stellar-Mass Black Holes}

\author{Will M. Farr \and Niharika Sravan \and Ilya Mandel} 

\affil{Northwestern University Center for Interdisciplinary
  Exploration and Research in Astrophysics\\2145 Sheridan Rd.,
  Evanston, IL 60208}

\email{w-farr@northwestern.edu, niharika.sravan@gmail.com}

\author{Andrew Cantrell \and Laura Kreidberg \and Charles Bailyn}

\affil{Yale University Department of Astrophysics\\
  P.O. Box 208101, New Haven, CT 06520}

\email{andrew.cantrell@yale.edu, laura.kreidberg@yale.edu,
  charles.bailyn@yale.edu}

\author{Vicky Kalogera} 

\affil{Northwestern University Center for Interdisciplinary
  Exploration and Research in Astrophysics\\2145 Sheridan Rd.,
  Evanston, IL 60208}

 \email{vicky@northwestern.edu}

\begin{abstract}
  We perform a Bayesian analysis of the mass distribution of
  stellar-mass black holes based on the observed masses of the compact
  object in 17 low mass X-ray binaries.  Modeling the mass
  distribution both parametrically---as a power law, exponential,
  gaussian, or combination of two gaussians---and
  non-parametrically---as histograms with varying numbers of
  bins---we explore the distribution in the context of Markov Chain
  Monte Carlo calculations of model parameters.  We give confidence
  bounds on the shape of the mass distribution in the context of each
  model and compare the models with each other by calculating their
  Bayesian evidence.  We address the existence of a ``gap'' between
  the most massive neutron stars and least massive black holes,
  finding that the best model (the power law), the second-best model
  (a one-bin histogram, or flat distribution), and the third-best
  model (a two-bin histogram) have minimum black hole masses above
  5.00, 3.23, and 3.82 solar masses (95\% confidence), respectively.
  We therefore conclude that our sample of black hole masses provides
  strong evidence of a gap between the maximum neutron star mass and
  minimum black hole mass.
\end{abstract}

\maketitle

\section{Introduction}
\label{sec:intro}

\section{Systems}
\label{sec:systems}

The 17 low mass X-ray binary systems on which this study is based are
listed in Table \ref{tab:sources}.  In each of these systems,
spectroscopic measurements of the secondary star provide an orbital
period for the system and a half-amplitude for the secondary's
velocity curve.  These measurements can be combined into the mass
function,
\begin{equation}
  \label{eq:mass-function}
  f(M) = \frac{P K^3}{2\pi G} = \frac{M \sin^3 i}{\left( 1 + q \right)^2},
\end{equation}
where $P$ is the orbital period, $K$ is the secondary's velocity
semi-amplitude, $M$ is the black hole mass, $i$ is the inclination of
the system, and $q \equiv M_2 / M$ is the mass ratio of the system.

The mass function defines a lower limit on the mass: $f(M) < M$.  To
accurately determine the mass of the black hole, the inclination $i$
and mass ratio $q$ must be measured.  Ideally, this can be
accomplished by fitting ellipsoidal light curves from these systems
but even in the ideal case (see, e.g., \citet{Cantrell2010} on A0620)
this procedure can be complicated by possible contributions from an
accretion disk and hot spots in the disk.  For some systems (e.g.\ GS
1354 \citep{Casares2009}) strong variability prevents determination of
the inclination from the lightcurve at all; in these cases an upper
limit on the inclination often comes from the observed lack of
eclipses in the lightcurve.  In general, accurately determining $q$
and $i$ requires a careful system-by-system analysis.

For the purposes of this paper, we adopt the following simplified
approach to the estimation of the black hole mass from the observed
data.  When an observable is well-constrained, we assume that the true
value is normally distributed about the measured value with a standard
deviation equal to the quoted observational error.  This is the case
for the mass function in all the systems we use, and for many systems'
mass ratios and inclinations.  When a large range is quoted in the
literature for an observable, we take the true value to be distributed
uniformly (for the mass ratio) or isotropically (for the inclination)
within the quoted range.  Table \ref{tab:sources} gives the assumed
distribution for the observables in the 17 systems we use.  We do not
attempt to deal with the systematic biases in the observational
determination of $f$, $q$, and $i$ in any realistic way, instead
leaving this for future work.

\begin{table}
  \begin{center}
    \begin{tabular}{|l|c|c|c|l|}
      \hline
      Source & $f$ ($\Msun$) & $q$ & $i$ (degrees) & References \\
      \hline \hline
      GRS 1915 & $N(9.5, 3.0)$ & $N(0.0857, 0.0284)$ & $N(70, 2)$ &
      \citet{Greiner2001} \\
      XTE J1118 & $N(6.44, 0.08)$ & $N(0.0264, 0.004)$ & $N(68, 2)$ &
      \citet{Gelino2008} \\ & & & & \citet{Harlaftis2005} \\
      Cyg X1 & $N(0.251, 0.007)$ & $N(2.778, 0.386)$ & $I(23, 38)$ &
      \citet{Gies2003} \\ 
      XTE J1650 & $N(2.73, 0.56)$ & $U(0, 0.5)$ & $I(50, 80)$ &
      \cite{Orosz2004} \\
      GRS 1009 & $N(3.17, 0.12)$ & $N(0.137, 0.015)$ & $I(37, 80)$ &
      \cite{Filippenko1999} \\
      M33 X7 & $N(0.46, 0.08)$ & $N(4.47, 0.61)$ & $N(74.6, 1)$ &
      \citet{Orosz2007} \\
      A0620 & $N(2.76, 0.036)$ & $N(0.06, 0.004)$ & $N(50.98, 0.87)$ &
      \citet{Cantrell2010} \\ & & & & \citet{Neilsen2008} \\
      GRO J0422 & $N(1.13, 0.09)$ & $U(0.076, 0.31)$ & $N(45, 2)$ &
      \citet{Gelino2003} \\
      Nova Mus 1991 & $N(3.01, 0.15)$ & $N(0.128, 0.04)$ & $N(54,1.5)$
      & \cite{Gelino2001} \\
      GRO J1655 & $N(2.73,0.09)$ & $N(0.3663, 0.04025)$ & $N(70.2,
      1.9)$ & \citet{Greene2001} \\
      4U 1543 & $N(0.25, 0.01)$ & $U(0.25, 0.31)$ & $N(20.7,1.5)$ & 
      \citet{Orosz2003} \\
      XTE J1550 & $N(7.73,0.4)$ & $U(0,0.04)$ & $N(74.7, 3.8)$ &
      \citet{Orosz2010} \\
      V4641 Sgr & $N(3.13,0.13)$ & $U(0.42,0.45)$ & $N(75,2)$ &
      \citet{Orosz2003} \\
      GS 2023 & $N(6.08, 0.06)$ & $U(0.056,0.063)$ & $N(55,4)$ &
      \citet{Charles2006} \\
      GS 1354 & $N(5.73, 0.29)$ & $N(0.12,0.04)$ & $I(50, 80)$ & 
      \citet{Casares2009} \\
      Nova Oph 77 & $N(4.86,0.13)$ & $U(0, 0.053)$ & $I(60, 80)$ &
      \citet{Charles2006} \\
      GS 2000 & $N(5.01, 0.12)$ & $U(0.035, 0.053)$ & $I(43, 74)$ &
      \citet{Charles2006} \\
      \hline
    \end{tabular}
  \end{center}
  \caption{\label{tab:sources} The source parameters for the 17 low
    mass X-ray binaries used in this work.  $f$ is the mass function for
    the compact object, $q$ is the mass ratio $M_2/M$, and $i$ is the
    inclination of the system to the line of sight.  We indicate the
    distribution used for the true parameters when computing the
    probability distributions for the masses of these systems:
    $N(\mu,\sigma)$ implies a Gaussian with mean $\mu$ and standard
    deviation $\sigma$, $U(a,b)$ is a uniform distribution between $a$ and
    $b$, and $I(\alpha,\beta)$ is an isotropic distribution between the
    angles $\alpha$ and $\beta$.  \fixme{Orosz for XTE J1550 needs a
      real reference \citep{Orosz2010}.}} 
\end{table}

From these assumptions, we can generate probability distributions for
the true mass of the black hole given the observations and errors via
the Monte Carlo method: drawing samples of $f$, $q$, and $i$ from the
assumed distributions and computing the mass implied by Equation
\eqref{eq:mass-function} gives samples of $M$ from the distribution
induced by the relationship in Equation \eqref{eq:mass-function}.
Mass distributions generated in this way for the 17 systems used in
this work are shown in Figure \ref{fig:all-masses}.  These mass
distributions constitute the ``observational data'' we will use in
Section \ref{sec:models} to explore various underlying mass
distributions.

\begin{figure}
  \begin{center}
    \plotone{plots/all-masses} 
  \end{center}

  \caption{\label{fig:all-masses} The mass distributions implied by
    Equation \eqref{eq:mass-function} and the assumed distributions on
    observational parameters given in Table \ref{tab:sources}.  The
    significant asymmetry and long tails in many of these
    distributions are the result of the non-linear relationship
    (Equation \eqref{eq:mass-function}) between $M$, $f$, $q$, and
    $i$.}
\end{figure}

In Figure \ref{fig:masses} we show the combination of these individual
mass distributions into one single distribution.  (Note that this
combination is not a good estimator of the true mass distribution of
black hole masses---see, e.g., \citet{Mandel2010}.  We include it here
only to give a sense of the range of masses of black holes in our
sample; we will come to estimators for the mass distribution.)

\begin{figure}
  \plotone{plots/masses}

  \caption{\label{fig:masses} The mass distributions in Figure
    \ref{fig:all-masses} combined into a single distribution. Note
    that this combination is not a good estimator of the true mass
    distribution of black hole masses in the galaxy---see, e.g.,
    \citet{Mandel2010}.  We include it here only to give a sense of
    the range and general shape of the mass distribution of black
    holes in our sample.}
\end{figure}

\section{Models}
\label{sec:models}

In this section we apply a Bayesian analysis to various models for the
underlying mass distribution from which the systems in Table
\ref{tab:sources} were drawn.  The end result will be the probability
distribution for the parameters of each model implied by the data from
Section \ref{sec:systems} in combination with our prior assumptions
about the probability distribution for the parameters.  Bayes' rule
relates these quantities.  For a model with parameters $\theta$ in the
presence of data $d$, Bayes' rule states
\begin{equation}
  \label{eq:Bayes-rule}
  p(\theta | d) = \frac{p(d | \theta) p(\theta)}{p(d)}.
\end{equation}
Here, $p(\theta|d)$, called the posterior probability distribution
function, is the probability distribution for the parameters $\theta$
implied by the data $d$; $p(d|\theta)$, called the likelihood, is the
probability of observing data $d$ given that the model parameters are
$\theta$; $p(\theta)$, called the prior, reflects our estimate of the
probability of the various model parameters in the absence of any
data; and $p(d)$, called the evidence, is an overall normalizing
constant ensuring that 
\begin{equation}
  \int d\theta\, p(\theta|d) = 1,
\end{equation}
whence
\begin{equation}
  \label{eq:evidence-def}
  p(d) = \int d\theta\, p(d|\theta) p(\theta).
\end{equation}

In our context, the data are the mass distributions given in Section
\ref{sec:systems}: $d = \{ p_i(m)| i = 1, 2, \ldots, 17 \}$.  We
assume that the measurements in Section \ref{sec:systems} are
independent, so the complete likelihood is given by a product of the
likelihoods for the individual measurements.  For a model with
parameters $\theta$ that predicts a mass distribution $p(m|\theta)$
for black holes, we have
\begin{equation}
  \label{eq:likelihood-def}
  p(d|\theta) = \prod_i \int dm\, p_i(m) p(m|\theta).
\end{equation}
That is, the likelihood of an observation is the average over the
individual mass distribution implied by the observation, $p_i(m)$, of
the probability for a black hole of that mass to exist according to
the model of the mass distribution, $p(m | \theta)$.  We approximate
the integrals as averages of $p(m|\theta)$ over the Monte Carlo mass
samples drawn from the distributions in Table \ref{tab:sources} (also
see Figure \ref{fig:all-masses}):
\begin{equation}
  p(d|\theta) \approx \prod_i \frac{1}{N_i} \sum_{j = 1}^{N_i} p(m_{ij} | \theta),
\end{equation}
where $m_{ij}$ is the $j$th sample (out of a total $N_i$) from the
$i$th individual mass distribution.

Our calculation of the likelihood of each observation does not include
any attempt to account for selection effects in the observations.  We
simply assume (almost certainly incorrectly) that any black hole drawn
from the underlying mass distribution is equally likely to be
observed, leaving the analysis of selection effects to future work.

For a mass distribution with several parameters, $p(\theta | d)$ lives
in a multi-dimensional space.  Exploring the entirety of this space
systematically rapidly becomes prohibitive as the number of parameters
increases.  A more efficient way to explore the distribution $p(\theta
| d)$ is to use a Markov Chain Monte Carlo (MCMC) method.  MCMC
methods produce a chain (sequence) of parameter samples, $\{ \theta_i,
i = 1, \ldots \}$, such that a particular parameter, $\theta$, appears
in the sequence with a frequency equal to its probability according to
$p(\theta|d)$.  In this way, regions of parameter space where
$p(\theta|d)$ is large are sampled densely while regions where
$p(\theta|d)$ is small are effectively ignored.  A Markov chain has
the property that the transition probability from one element to the
next, $p(\theta_i \to \theta_{i+1})$, depends only on the value of
$\theta_i$, not on any previous values in the chain.

One way to produce a sequence of MCMC samples is via the following
algorithm, first proposed by \fixme{Metropolis citation needed}:
\begin{enumerate}
  \item Begin with the current sample, $\theta_i$.
  \item Propose a new sample, $\theta_p$, by drawing randomly from an
    arbitrary ``jump proposal distribution'' with probability
    $Q(\theta_i \to \theta_p)$.
  \item Compute the ``acceptance'' probability,
    \begin{equation}
      \label{eq:paccept}
      p_{\textnormal{accept}} \equiv
      \frac{p(\theta_p|d)}{p(\theta_i|d)} \frac{Q(\theta_p \to
        \theta_i)}{Q(\theta_i \to \theta_p)}
    \end{equation}
  \item With probability $\min(1,p_{\textnormal{accept}})$ set
    $\theta_{i+1} = \theta_p$; otherwise set $\theta_{i+1} =
    \theta_i$.
\end{enumerate}
Note that this algorithm is more likely to accept a proposed jump when
it increases the posterior (the first term in Equation
\eqref{eq:paccept}) and when it is to a location in parameter space
from which it is easy to return (the second term in Equation
\eqref{eq:paccept}).  As $i \to \infty$ the samples $\theta_i$ are
distributed according to $p(\theta|d)$.  In practice the number of
samples required before the chain appropriately samples $p(\theta|d)$
depends strongly on the jump proposal distribution; proposal
distributions that often propose jumps toward or within regions of
large $p(\theta|d)$ can be very efficient, while poor proposal
distributions can require prohibitively large numbers of samples
before convergence.  

Once we have a chain of samples from $p(\theta|d)$---a distribution of
distributions---the distribution for any quantity of interest can be
computed.  For example, Figure \ref{fig:dists} shows the median
values, and the 10\%, and 90\% quantiles for value of the BH mass
probability distribution at various masses in each of the models
discussed in the following subsections.  Such a figure can be
generated from the MCMC output by computing the value of the
parameterized BH mass distribution at various masses for each
parameter sample output by the MCMC, and then plotting the quantiles
at each mass.

\begin{figure}
  \begin{center}
    \plottwo{plots/dist-parameteric}{plots/dist-non-parameteric}
  \end{center}
  \caption{\label{fig:dists} The median values of the black hole mass
    distribution, $p(m|\theta)$, at various masses implied by the
    posterior $p(\theta|d)$ for the models discussed in Sections
    \ref{subsec:parameteric-models} and
    \ref{subsec:non-parameteric-models}.  Error bars span the 10\% to
    90\% range.  Note that these ``distributions of distributions''
    are not necessarily normalized, and need not be ``shaped'' like
    the underlying model distributions.}
\end{figure}

It is also common to look at the one-dimensional distribution for a
single parameter obtained by integrating over all other dimensions in
parameter space; this is approximated by a histogram of the MCMC
sample values for that parameter.  Such a distribution is called the
``marginalized'' distribution.

We defer discussion of the appropriateness of the different models
given the data (i.e.\ the ``goodness of fit'') until Section
\ref{sec:model-selection}.

\subsection{Parameteric Models for the Black Hole Mass Distribution}
\label{subsec:parameteric-models}

In this subsection, we discuss the various parameteric models of the
underlying black hole mass distribution that we have analyzed using
the data from Section \ref{sec:systems}.

\subsubsection{Power-Law Models}
\label{subsubsec:power-law}

Many astrophysical distributions are power-laws.  Let us assume that
the BH mass distribution is given by
\begin{equation}
  \label{eq:power-law-dist}
  p(m|\theta) = p(m|\{\Mmin, \Mmax, \alpha \}) =
  \begin{cases}
    A m^\alpha & \Mmin \leq m \leq \Mmax \\
    0 & \textnormal{otherwise}
  \end{cases}.
\end{equation}
The normalizing constant $A$ is given by 
\begin{equation}
  A = \frac{1+\alpha}{\Mmax^{1+\alpha} - \Mmin^{1+\alpha}}.
\end{equation}
To perform the Bayesian analysis, we need to impose a prior on the
parameters $\Mmin$, $\Mmax$, $\alpha$.  We choose to impose a broad
prior that represents our significant uncertainty in the appropriate
values of these parameters.  We use
\begin{equation}
  p(\theta) = p(\{\Mmin, \Mmax, \alpha\}) = 
  \begin{cases}
    2 \frac{1}{40^2}\frac{1}{20} & 0 \leq \Mmin \leq \Mmax
    \leq 40, \quad -12 \leq \alpha \leq 8 \\
    0 & \textnormal{otherwise}
  \end{cases},
\end{equation}
where the masses are measured in solar masses.  This prior is uniform
in the masses between 0 and $40 \Msun$, and uniform in power-law slope
between $-12$ and $8$.

Our MCMC analysis output is a list of $\{\Mmin, \Mmax, \alpha\}$
values distributed according to the posterior 
\begin{equation}
  p(\theta|d) = p(\{\Mmin, \Mmax, \alpha\}|d) \propto p(d|\{\Mmin,
  \Mmax, \alpha\}) p(\{\Mmin, \Mmax, \alpha\}),
\end{equation}
with the likelihood $p(d|\{\Mmin, \Mmax, \alpha\})$ defined in
Equation \eqref{eq:likelihood-def}.  Figure \ref{fig:dists} presents
the resulting distribution of mass distributions inferred from the
MCMC distribution of parameters.

In Figure \ref{fig:alpha}, we display a histogram of the resulting
samples in $\alpha$; this represents the one-dimensional
``marginalized'' distribution
\begin{equation}
  \label{eq:alpha-pdf}
  p(\alpha|d) = \int d\Mmin\, d\Mmax\, p(\{\Mmin, \Mmax, \alpha\}|d).
\end{equation}
The resulting distribution is quite broad with 
\begin{equation}
  -7.66 < \alpha < -1.36
\end{equation}
enclosing 90\% of the probability.  The median value is $\alpha =
-4.28$, and we have $\alpha < 0$ (a falling underlying mass
distribution) at 98.7\% confidence.  The distribution of the minimum
mass is also quite interesting, but we defer this discussion until
Section \ref{sec:minimum-mass}, where we discuss the distribution for
the minimum black hole masses in the context of all our models.

\begin{figure}
  \begin{center}
    \plotone{plots/alpha}
  \end{center}
  \caption{\label{fig:alpha} A histogram of the MCMC samples for
    $\alpha$ from the power-law distribution defined in Equation
    \eqref{eq:power-law-dist}.  The histogram is an approximation to
    the one-dimensional marginalized PDF in Equation
    \eqref{eq:alpha-pdf}.  The favored value is near $\alpha = -4$.
    The fraction of the distribution with $\alpha < 0$ is 98.7\%.  The
    5\% quantile is $-7.66$, the median is $-4.28$, and the 95\%
    quantile is $-1.36$.}
\end{figure}

\subsubsection{Decaying Exponential}
\label{subsubsec:exponential}

Theoretical arguments in \citet{Fryer2001} suggest that the black-hole
mass distribution may be well-represented by a decaying exponential
with a minimum mass:
\begin{equation}
  \label{eq:exp-def}
  p(m|\theta) = p(m|\{\Mmin, M_0\}) = 
  \begin{cases}
    \frac{e^{\frac{\Mmin}{M_0}}}{M_0} \exp \left[ - \frac{m}{M_0}
    \right] & \Mmin \leq m \\
    0 & \textnormal{otherwise}
  \end{cases}.
\end{equation}
We choose again a broad, flat prior
\begin{equation}
  p(\{\Mmin, M_0 \}) = 
  \begin{cases}
    \frac{1}{40^2} & 0 \leq \Mmin, M_0 \leq 40 \\
    0 & \textnormal{otherwise}
  \end{cases},
\end{equation}
with $\Mmin$ and $M_0$ measured in solar masses.

Figure \ref{fig:exp-M0} displays the marginalized posterior
distribution for the scale mass of the exponential, $M_0$.  The median
scale mass is $M_0 = 3.00$, and $1.57 \leq M_0 \leq 5.71$ with 90\%
confidence.  We defer discussion of the distribution of $\Mmin$ until
Section \ref{sec:minimum-mass}.  Refer to Figure \ref{fig:dists}
for the resulting shape of the distribution of distributions.

\begin{figure}
  \begin{center}
    \plotone{plots/exp-M0}
  \end{center}
  \caption{\label{fig:exp-M0} The distribution of scale masses, $M_0$,
    measured in units of a solar mass for the exponential underlying
    mass distribution defined in Equation \eqref{eq:exp-def}.  The
    median scale mass is $M_0 = 3.00$; with 90\% confidence $1.57 \leq
    M_0 \leq 5.71$. }
\end{figure}

\subsubsection{Gaussian and Two-Gaussian Models}
\label{subsubsec:gaussian}

The prototypical single-peaked probability distribution is a
Gaussian:
\begin{equation}
  p(m|\theta) = p(m|\{\mu, \sigma\}) = \frac{1}{\sigma \sqrt{2\pi}}
  \exp\left[ - \left(\frac{m - \mu}{\sqrt{2} \sigma} \right)^2 \right].
\end{equation}
We choose a broad, flat prior 
\begin{equation}
  \label{eq:gaussian-def}
  p(\{\mu,\sigma\}) = 
  \begin{cases}
    \frac{1}{40^2} & 0 \leq \mu, \sigma \leq 40 \\
    0 & \textnormal{otherwise}
  \end{cases},
\end{equation}
where both $\mu$ and $\sigma$ are measured in solar masses.  Figure
\ref{fig:gaussian} shows the resulting marginalized distributions for
the parameters $\mu$ and $\sigma$.  We constrain $\mu$ in $7.02 \leq
\mu \leq 9.23$ with 90\% confidence.  (Refer to Figure \ref{fig:dists}
for the shape of the resulting distribution of distributions.)

\begin{figure}
  \begin{center}
    \plotone{plots/gaussian}
  \end{center}
  \caption{\label{fig:gaussian} Marginalized posterior distributions
    for the mean $\mu$ and standard deviation $\sigma$ (both in solar
    masses) for the Gaussian underlying mass distribution defined in
    Equation \eqref{eq:gaussian-def}.  The peak of the Gaussian,
    $\mu$, is constrained in $7.02 \leq \mu \leq 9.23$ with 90\%
    confidence.}
\end{figure}

To look for a second peak in the black-hole mass distribution, we also
examined a two-Gaussian model:
\begin{multline}
  p(m|\theta) = p(m|\{\mu_1, \mu_2, \sigma_1, \sigma_2, \alpha\}) = \\
  \frac{\alpha}{\sigma_1 \sqrt{2\pi}} \exp\left[ - \left( \frac{m -
        \mu_1}{\sqrt{2}\sigma_1} \right)^2 \right] + \frac{1-\alpha}{\sigma_2 \sqrt{2\pi}} \exp\left[ - \left( \frac{m -
        \mu_2}{\sqrt{2}\sigma_2} \right)^2 \right],
\end{multline}
where we restrict $\mu_1 < \mu_2$ with our broad, flat prior:
\begin{equation}
  p(\{\mu_1, \mu_2, \sigma_1, \sigma_2, \alpha\}) = 
  \begin{cases}
    2 \frac{1}{40^4} & 0 \leq \mu_1 \leq \mu_2 \leq 40,
    \quad 0 \leq \sigma_1, \sigma_2 \leq 40, \quad 0 \leq \alpha \leq
    1 \\
    0 & \textnormal{otherwise}
  \end{cases}.
\end{equation}
The one-Gaussian model is a subset of the Gaussian model: when
$\alpha$ is small, then the presence of the first Gaussian is
irrelevant, and the second gives the same mass distribution as a
single Gaussian; when $\alpha$ is large, the reverse situation again
reproduces the single-Gaussian model; and when $\alpha$ is in the
middle of its range, the two means and standard deviations can be
nearly equal, reproducing again the single-Gaussian model.  This is
precisely what we observe from our MCMC samples in the two-Gaussian
model.  In Section \ref{sec:model-selection}, we will quantify more
precisely how much redundancy we have in this model relative to the
one-Gaussian model, but we should emphasize here that there is no
evidence of a significant contribution from a second peak in the
two-Gaussian model.

\subsection{Non-Parameteric Models for the Black Hole Mass Distribution}
\label{subsec:non-parameteric-models}

The previous subsection discussed models for the underlying black hole
mass distribution that assumed particular parameterized shapes for the
distribution.  In this subsection, we will discuss models that do not
assume a priori a shape for the black hole mass distribution.  The
fundamental non-parameteric distribution in this section is a
histogram with some number of bins, $\Nbin$.  Such a distribution is
piecewise-constant in $m$.

One choice for representing such a histogram would be to fix the bin
locations, and allow the heights to vary.  With this approach, one
should be careful not to ``split'' features of the mass distribution
across more than one bin in order to avoid diluting the sensitivity to
such features; similarly, one should avoid including more than ``one''
feature in each bin.  The locations of the bins, then, are crucial.
An alternative representation of histogram mass distributions avoids
this difficulty.

We choose to represent a histogram mass distribution with $\Nbin$ bins
by allocating a fixed total probability, $1/\Nbin$, to each bin.  The
lower and upper bounds for each bin are allowed to vary; when these
are close to each other (i.e.\ the bin is narrow), the distribution
will have a large value, and conversely when the bounds are far from
each other.  We assume that the non-zero region of the distribution is
contiguous, so we can represent the boundaries of the bins as a
non-decreasing array of masses, $w_0 \leq w_1 \leq \ldots \leq
w_{\Nbin}$, with $w_0$ the minimum and $w_{\Nbin}$ the maximum mass for
which the distribution has support.  This gives the distribution
\begin{equation}
  \label{eq:hist-def}
  p(m|\theta) = p(m|\{w_0, \ldots, w_{\Nbin}\}) = 
  \begin{cases}
    0 & m < w_0 \textnormal{ or } w_{\Nbin} \leq m \\
    \frac{1}{\Nbin} \frac{1}{w_{i+1} - w_i} & w_i \leq m < w_{i+1}
  \end{cases}.
\end{equation}

For priors on the histogram model with $\Nbin$ bins, we assume that
the bin boundaries are uniformly distributed between 0 and $40 \Msun$
subject only to the constraint that the boundaries are non-decreasing
from $w_0$ to $w_{\Nbin}$:
\begin{equation}
  p(\{w_0, \ldots, w_{\Nbin}\}) = 
  \begin{cases}
    \frac{\left(\Nbin+1\right)!}{40^{\Nbin+1}} & 0 \leq w_0 \leq w_1
    \leq \ldots \leq w_{\Nbin} \leq 40 \\
    0 & \textnormal{otherwise}
  \end{cases}.
\end{equation}

The median values of the histogram mass distributions that result from
the MCMC samples of the posterior distribution for the $w_i$
parameters for one-, two-, three-, four-, and five-bin histogram
models are shown in Figure \ref{fig:dists}.  As the number of bins
increases, the models are better able to capture features of the mass
distribution.  (But note that in Section \ref{sec:model-selection}, we
will find that the one-bin histogram---which is a flat distribution
between a maximum and minimum mass---is the best model from this
group.)  We defer discussion of the distribution of the minimum mass
from the histogram distributions until Section \ref{sec:minimum-mass}.

\section{Model Selection}
\label{sec:model-selection}

In Section \ref{sec:models}, we discussed a series of models for the
underlying black hole mass distribution.  For each model, we have
assumed that the underlying mass distribution corresponds to the
model, and we ask what distributions are implied by the data for the
parameters in the model.  We have not yet asked which models are more
likely to correspond to the actual distribution.  That is the topic of
this section.

To compare models in the context of a Bayesian analysis, consider a
set of models, $\{M_i| i = 1, \ldots\}$, each with corresponding
parameters $\theta_i$.  Re-writing Equation \eqref{eq:Bayes-rule} to
be explicit about the assumption of a particular model, we have
\begin{equation}
  \label{eq:bayes-explicit-model}
  p(\theta_i | d, M_i) = \frac{p(d|\theta_i, M_i) p(\theta_i | M_i)}{p(d|M_i)}.
\end{equation}
This gives the posterior probability of the parameters $\theta_i$ in
the context of model $M_i$.  But, the particular model itself can be
regarded as a parameter in a larger ``super-model'' that encompasses
all the $M_i$.  If we let the model become a parameter, Bayes' rule
gives us
\begin{equation}
  p(\theta_i, M_i|d) = \frac{p(d|\theta_i, M_i) p(\theta_i |M_i) p(M_i)}{p(d)},
\end{equation}
where we have introduced the model prior $p(M_i)$, which represents
our estimate on the probability that model $M_i$ is correct in the
absence of the data $d$.  The normalizing evidence is now
\begin{equation}
  p(d) = \sum_i \int d\theta_i\, p(\theta_i, M_i|d).
\end{equation}

To compare the various models $M_i$, we are interested in the
marginalized posterior probability of $M_i$ (also called the evidence
for model $M_i$):
\begin{equation}
  p(M_i|d) \equiv \int d\theta_i\, p(\theta_i, M_i|d).
\end{equation}
This quantity can be re-written in terms of the single-model evidence
(see Equations \eqref{eq:bayes-explicit-model} and
\eqref{eq:evidence-def}):
\begin{equation}
  \label{eq:model-evidence-def}
  p(M_i|d) = \frac{p(M_i)}{p(d)} \int d\theta_i
  p(d|\theta_i,M_i) p(\theta_i|M_i) = \frac{p(d|M_i) p(M_i)}{p(d)}.
\end{equation}

There are two broad ways to compare models in the context of the MCMC
analysis of Section \ref{sec:models}.  The first way, which we discuss
in Section \ref{subsec:direct-evidence}, is to use the MCMC samples
from the various models to estimate the value of the integral in
Equation \eqref{eq:model-evidence-def}.  The second way is to treat
the model literally as a parameter in a second MCMC analysis.  One
then proposes jumps (see the discussion in Section \ref{sec:models})
not only for the $\theta_i$ within a particular model, but from
parameters $\theta_i$ in model $M_i$ to parameters $\theta_j$ in
$M_j$.  This technique is called a ``reversible-jump MCMC''
\citep{Green1995}; we discuss it in Section
\ref{subsec:reversible-jump-mcmc}.

\subsection{Direct Evidence Calculations}
\label{subsec:direct-evidence}

In this section we discuss techniques for using the MCMC samples from
Section \ref{sec:models} to estimate the integral in Equation
\eqref{eq:model-evidence-def}.  The first technique we use comes from
re-writing Equation \eqref{eq:bayes-explicit-model}:
\begin{equation}
  \frac{p(\theta_i|d,M_i)}{p(d|\theta_i,M_i)} = \frac{p(\theta_i|M_i)}{p(d|M_i)}.
\end{equation}
Integrating both sides over all $\theta_i$, recalling that the prior
is normalized, we have 
\begin{equation}
  \int d\theta_i\, \frac{p(\theta_i|d,M_i)}{p(d|\theta_i,M_i)} = \frac{1}{p(d|M_i)}.
\end{equation}
We can estimate the integral using our MCMC samples (which are
distributed with local density $p(\theta_i|d,M_i)d\theta_i$) via
\begin{equation}
  \int d\theta_i\, \frac{p(\theta_i|d,M_i)}{p(d|\theta_i,M_i)} \approx
  \frac{1}{N_i} \sum_j \frac{1}{p(d|\theta_{ij}, M_i)},
\end{equation}
where the $j$th MCMC parameter sample (from a total $N_i$) is
$\theta_{ij}$.  Thus we have
\begin{equation}
  \frac{1}{p(d|M_i)} \approx \frac{1}{N_i} \sum_j \frac{1}{p(d|\theta_{ij}, M_i)};
\end{equation}
the evidence for model $M_i$ can be estimated from the harmonic mean
of the likelihood over the MCMC samples.  

Alternately, we can attempt to perform the integral over $\theta_i$
for the model evidence directly using our MCMC samples.  If we could
assign non-overlapping volumes, $V_j$, to each of the $N_i$ MCMC
samples, $\theta_{ij}$, then we could approximate the integral in
Equation \eqref{eq:model-evidence-def} via
\begin{equation}
  \label{eq:direct-evidence-estimate}
  \int d\theta_i
  p(d|\theta_i,M_i) p(\theta_i|M_i) \approx \sum_j
  p(d|\theta_{ij},M_i) p(\theta_{ij}|M_i) V_j.
\end{equation}

To associate a volume to each MCMC sample, we use a data structure
known as a kD-tree.  (Our algorithm is similar to Algorithm P of
\citet{Weinberg2010}.)  A kD-tree decomposes a volume in
$\mathbb{R}^n$ into non-overlapping boxes, each of which contains some
number of points from a given sample.  The decomposition proceeds
recursively: given a region to subdivide and a sample of points, a
dimension is chosen (we choose the dimension of largest extent of the
sample, but other choices are possible) and the region is split along
this dimension such that the numbers of points to each side of the
partition are equal or differ by one.  Each of the resulting
sub-regions and the sub-samples contained in them are further split in
the same manner, until every region contains fewer than some chosen
number of points.  The sequence of splittings forms a binary tree,
with the initial region and point sample at the root, and the final
boxes and subsamples at the leaves.  From experimentation on
artificial point distributions, we find that terminating the splitting
when each box contains fewer than $\sim 64$ to $\sim 256$ samples is
reasonably effective in estimating evidence integrals in $\sim 5$
dimensions.  (See also \citet{Weinberg2010}.)  To estimate the
integral as in Equation \eqref{eq:direct-evidence-estimate}, we walk
the leaves of the kD-tree; for each leaf, we compute the volume
spanned by the samples in that leaf, and assign an equal fraction of
this volume to each sample (note that the volume spanned by the
samples is \emph{smaller} than the leaf region itself).  The integral
is computed using the assigned volumes as in Equation
\eqref{eq:direct-evidence-estimate}.

The harmonic mean estimate of the evidence is known for having large
statistical uncertainty due to the tendency for a few low-likelihood
points to dominate the sum \citep{Weinberg2010}; the statistics of the
kD-tree estimate are less well-known.  In any case, it is useful to
have multiple independent estimates of the evidence for a given model.
In Figure \ref{fig:direct-evidence} we use both methods to estimate
the evidence for the various models discussed in Section
\ref{sec:models}.  We assume that no model is a priori more likely
than any other, so the model priors $p(M_i)$ are equal.  We also
ignore the model-independent constant $p(d)$ in Equation
\eqref{eq:model-evidence-def}, so only ratios of evidence are
relevant.  Due to the large statistical uncertainty expected in the
harmonic mean estimate of the evidence, we perform bootstrap estimates
of the range of the harmonic mean, reporting the 10\% and 90\% values
in the error bars of Figure \ref{fig:direct-evidence}.

\begin{figure}
  \begin{center}
    \plotone{plots/evidence}
  \end{center}
  \caption{\label{fig:direct-evidence} Direct integration and harmonic
    mean evidence for the various models from Section \ref{sec:models}
    (see Section \ref{subsec:direct-evidence}).  In increasing
    order along the $x$-axis, the models are the power-law of
    Section \ref{subsubsec:power-law} (PL), the decaying
    exponential of Section \ref{subsubsec:exponential} (E), the
    single Gaussian of Section \ref{subsubsec:gaussian} (G), the
    double Gaussian of Section \ref{subsubsec:gaussian} (TG),
    and the one-, two-, three-, four-, and five-bin histogram models
    of Section \ref{subsec:non-parameteric-models} (H1, H2, H3, H4,
    H5, respectively).  The cross symbols give the value of the direct
    integration; the plus symbols give the harmonic mean estimator
    with bootstrap estimates of the 10\% and 90\% bounds on the
    harmonic mean.  It appears that the power law and one-bin
    histogram are the two best models, but the error ranges and
    inconsistency between the two techniques for calculating evidence
    cast some doubt on this conclusion.}
\end{figure}

The results of the direct evidence calculation appear to point to the
power law and one-bin histogram as the two most favored models, but
the error ranges and inconsistency between the two techniques for
calculating evidence cast some doubt on this conclusion.  In the next
section, we will confirm these results with a more accurate method for
estimating the evidence.

\subsection{Reversible-Jump MCMC}
\label{subsec:reversible-jump-mcmc}

The technique of reversible-jump MCMC \citep{Green1995} follows the
``super-model'' concept, in which the choice of model, $M_i$, becomes
a parameter, to its logical conclusion.  One performs an MCMC that
proposes jumps between the parameter spaces of the various models that
are under consideration (i.e.\ from parameters $\theta_i$ in model
$M_i$ to parameters $\theta_j$ in model $M_j$).  For this MCMC to be
efficient, proposed jumps into a model should favor regions with large
posterior; when the posterior is highly-peaked in a small region of
parameter space, proposed jumps outside this region are unlikely to be
accepted, and the reversible-jump MCMC samples will require a very
long chain to properly sample the ``super-model'' posterior.  The
evidence for a particular model is proportional to the number of
reversible-jump MCMC samples that live in that model's parameter
space.

We use the kD-trees from the last subsection to interpolate the
posterior between MCMC samples from single-model MCMCs when proposing
jumps in our reversible-jump MCMC.  Thus, proposed jumps track the
posteriors computed from the single-model MCMCs.  (For an alternative
approach to generating jump proposals for reversible-jump MCMCs, see
\citet{Littenberg2009}.)  The interpolation algorithm works as
follows.  We construct a kD-tree as described in the last subsection,
continuing the subdivision until each region contains exactly one
sample point.  To propose a jump, we first choose a model uniformly
from the set of models (i.e.\ according to the model priors).  Then we
choose a point uniformly from the sample chain (i.e.\ a point is
chosen according to the posterior).  Then we find the leaf region in
the kD-tree that contains this point, and propose a jump uniformly
within this region.  This procedure for proposing a jump admits a
straightforward calculation of the jump proposal distribution.  The
advantage of using a kD-tree instead of the method in
\citet{Littenberg2009} is that the tree is locally adaptive to the
structure of the MCMC samples, and therefore more efficiently tracks
the posterior from each of the single-model MCMCs.

We have performed a reversible-jump MCMC that jumps between all the
models (both parameteric and non-parameteric) in Section
\ref{sec:models}.  The counts for the number of samples in each model
appear in Figure \ref{fig:rj}; relative probabilities of the models
are given in Table \ref{tab:rj}.  The most favored model is the power
law from Section \ref{subsubsec:power-law}, followed by the one-
and two-bin histograms from Section
\ref{subsec:non-parameteric-models}.  This confirms the results from
the direct integration and harmonic mean evidence estimators in the
last subsection.  Interestingly, the theoretical curve from
\citet{Fryer2001} places fourth in the ranking of evidence.  However,
while the exponential distribution is a factor of 3.9 times less
likely than the best-fit power law model, it is only 1.2 times less
likely than the two best histogram models.  A choice of model prior
that only slightly favored the exponential model over the
non-parameteric histograms would put it in second place.

\begin{table}
  \begin{center}
    \begin{tabular}{|l|r|}
      \hline
      Model & Relative Probability \\
      \hline \hline
      Power Law (Section \ref{subsubsec:power-law}) & 0.441660 \\
      \hline
      Histogram (1 Bin, Section \ref{subsec:non-parameteric-models}) &
      0.143435 \\
      \hline
      Histogram (2 Bin, Section \ref{subsec:non-parameteric-models}) &
      0.135854 \\
      \hline 
      Exponential (Section \ref{subsubsec:exponential}) & 0.112514 \\
      \hline
      Histogram (3 Bin, Section \ref{subsec:non-parameteric-models}) &
      0.075592 \\
      \hline
      Histogram (4 Bin, Section \ref{subsec:non-parameteric-models}) &
      0.038839 \\
      \hline 
      Gaussian (Section \ref{subsubsec:gaussian}) & 0.022999 \\ 
      \hline
      Histogram (5 Bin, Section \ref{subsec:non-parameteric-models}) &
      0.021011 \\
      \hline      
      Two Gaussian (Section \ref{subsubsec:gaussian}) & 0.008096 \\
      \hline
    \end{tabular}
  \end{center}
  \caption{\label{tab:rj} Relative probabilities of the various models
    from Section \ref{sec:models} computed from reversible-jump
    MCMC samples.}
\end{table}

Assuming that the reversible-jump MCMC has converged (and we have
tested ours for convergence), the standard deviation on the number of
counts in each model is given by
\begin{equation}
  \sigma_{N_i} = \sqrt{N_i \left( 1 - \frac{N_i}{N} \right)},
\end{equation}
where $N_i$ is the number of counts in model $M_i$, and $N$ is the
total number of reversible-jump counts.  This quantity is extremely
small relative to the statistical error in the harmonic mean estimator
or the likely truncation error on the direct integration estimate, so
we consider the reversible-jump evidence results to be definitive.

\begin{figure}
  \begin{center}
    \plotone{plots/rj}
  \end{center}
  \caption{\label{fig:rj} Counts of samples in the parameter space of
    each model from Section \ref{sec:models} in the reversible-jump
    MCMC of Section \ref{subsec:reversible-jump-mcmc}.  The models
    are the power law of Section \ref{subsubsec:power-law} (PL),
    the exponential of Section \ref{subsubsec:exponential} (E),
    the Gaussian (G) and two Gaussian (TG) models of Section
    \ref{subsubsec:gaussian}, and the one-, two-, three-, four-, and
    five-bin histograms of Section
    \ref{subsec:non-parameteric-models} (H1, H2, H3, H4, H5,
    respectively).  The number of counts in each model is proportional
    to the evidence for that model; therefore, the most favored model
    is the power-law, followed by the one- and two-bin histogram.  }
\end{figure}

\section{The Minimum Mass of the Black Hole Mass Distribution}
\label{sec:minimum-mass}

Using the MCMC samples for the parameters of the distributions in
Section \ref{sec:models}, we can form probability distributions for
the minimum black hole mass implied by the model posteriors.  (For the
Gaussian models without a sharp cutoff we use the mass corresponding
to the 1\% quantile of the distribution as the ``minimum black hole
mass.'')  Figure \ref{fig:min-mass} displays the probability
distributions for the minimum mass in the various models.  

\begin{figure}
  \begin{center}
    \plottwo{plots/mmin-parameteric}{plots/mmin-non-parameteric}
  \end{center}
  \caption{\label{fig:min-mass} The distributions for the minimum
    black hole mass calculated from the MCMC samples for the models in
    Section \ref{sec:models}.  For the Gaussian distributions without
    a sharp cutoff, we use the mass corresponding to the 1\% mass
    quantile for the ``minimum black hole mass.''  For the three
    most-favored models (the power law, one-, and two-bin histogram),
    the minimum black hole mass is above 5.00, 3.23, and 3.82 solar
    masses, respectively (at 95\% confidence). }
\end{figure}

For the three most-favored models (the power law, one-, and two-bin
histograms), the minimum black hole mass is above 5.00, 3.23, and 3.82
solar masses, respectively, with 95\% confidence.  While these limits
are lower than those of a previous study of fewer systems
\citep{Bailyn1998}, they are still significantly above the maximum
neutron star mass.  The measured masses of the systems in Section
\ref{sec:systems} show strong evidence of a ``gap'' between the
maximum neutron star and minimum black hole mass.

\section{Conclusion}

\acknowledgements

This work was supported by grant \fixme{XXX}.  Calculations for this
work were performed on the Northwestern Fugu cluster, which was
partially funded by NSF MRI grant \fixme{XXX}.  We thank Jonathan Gair
for helpful discussions.

\bibliography{paper}

\end{document}